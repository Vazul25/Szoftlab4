%\textit{% Szglab4
% ===========================================================================
%
\setcounter{chapter}{-1}
\chapter{Módosítások}
\comment{Feladatok: Objektum katalógus szövegeinek ellenőrzése, Osztályok leírásának pontjainak ellenőrzése, újonnan hozzáadott elemeinek felelősségének leírása}

\section{Követelmények}
\subsection{Funkcionális követelmények}

% Azonosító, Leírás, Ellenőrzés, Prioritás, Forrás, Use-case, Komment
\begin{longtable}{| l | l | l | l | l | l | l |}
\hline
\textbf{Azonosító}   & \textbf{Leírás} & \textbf{Ellenőrzés} & \textbf{Prioritás} & \textbf{Forrás} & \textbf{Use-case} & \textbf{Komment} \tabularnewline
\hline 1.01 & Játék menü & bemutatás & alapvető & csapat & &\tabularnewline
\hline 1.02 & Beállítások kezelése & bemutatás & opcionális & csapat & &\tabularnewline
\hline 1.03 & Játék elindítása & bemutatás & alapvető & megrednelő &\vtop{\hbox{\strut Új Játék}\hbox{\strut indítás}}&\tabularnewline
\hline 1.04 & Pályáról leesők kiesnek & bemutatás & alapvető & megrendelő & &\tabularnewline
\hline 1.05 & Előre elkészített versenypálya & bemutatás & alapvető & megrendelő & &\tabularnewline
\hline 1.06 & \vtop{\hbox{\strut Robotok a kezdőpozíciójukból}\hbox{\strut indulnak}}& bemutatás & alapvető & megrendelő & &\tabularnewline
\hline 1.07 &\vtop{\hbox{\strut Pályán vannak olajfoltok és}\hbox{\strut ragacsfoltok}}& bemutatás & alapvető & megrendelő & &\tabularnewline
\hline 1.08 &\vtop{\hbox{\strut Robotok fel vannak szerelve }\hbox{\strut olaj és ragacskészlettel}} & bemutatás & alapvető & megrendelő & &\tabularnewline
\hline 1.09 &\vtop{\hbox{\strut 2 személy tudjon játszani}\hbox{\strut egyszerre}} & bemutatás & alapvető & csapat & &\tabularnewline
\hline 1.10 &  Körlimites játékmód & bemutatás & opcionális & csapat & &\tabularnewline
\hline 1.11 &  Időlimites játékmód & bemutatás & alapvető & megrendelő & &\tabularnewline
\hline 1.12 & A robotok tudjanak ugrani & bemutatás & alapvető & megrendelő & &\tabularnewline
\hline 1.13 &\comment{...} & bemutatás & opcionális & \comment{...} & &\tabularnewline
\hline 1.14 & \vtop{\hbox{\strut A ragacs és olaj a pályáról,}\hbox{\strut ha belelépnek eltűnik\comment{...}}} & bemutatás & opcionális & \comment{...} & &\tabularnewline
\hline 1.15 & \vtop{\hbox{\strut A robotok tudnak egymással}\hbox{\strut ütközni}} & bemutatás & opcionális & \comment{...} & &\tabularnewline
\hline 1.16 &\vtop{\hbox{\strut Az indulás előtt}\hbox{\strut visszaszámlálás indul}} & bemutatás & opcionális & csapat & &\tabularnewline
\hline 1.17 &\vtop{\hbox{\strut A robotok sebessége }\hbox{\strut egységnyi méretű tetszőleges }\hbox{\strut irányú vektorral módosítható }} & bemutatás & alapvető & megrendelő & &\tabularnewline
\hline 1.18 &\vtop{\hbox{\strut Egy ugrással a sebességgel}\hbox{\strut egyenesen arányos}\hbox{\strut{távolságra tudnak eljutni}}} & bemutatás & alapvető & megrendelő & &\tabularnewline
\hline 1.19 &\vtop{\hbox{\strut A robot ragacsra érkezve}\hbox{\strut sebessége a felére csökken}} & bemutatás & alapvető & megrendelő & &\tabularnewline
\hline 1.20 &\vtop{\hbox{\strut A robot olajfoltra érkezve}\hbox{\strut sebességének módosítása}\hbox{nem lehetséges}} & bemutatás & alapvető & megrendelő & & \tabularnewline
\hline 1.21 &\vtop{\hbox{\strut A robot olajat vagy}\hbox{\strut ragacsot tudnak lerakni}} & bemutatás & alapvető & megrendelő & \vtop{\hbox{\strut Akadá-}\hbox{\strut lyozás}} & \tabularnewline
\hline 1.22 & & & alapvető & megrendelő & & \tabularnewline
\hline 1.23 & & & alapvető & megrendelő & & \tabularnewline
\hline 1.24 & & & alapvető & megrendelő & & \tabularnewline
\hline 1.25 & & & alapvető & megrendelő & & \tabularnewline

\hline
\end{longtable}

\section{Objektum katalógus}
\subsection{HUD}
Ez az objektum követi és nyilvántartja, hogy a robotok hány checkpoint-on mentek át, illetve kiírja a képernyőre a hátramaradó időt és a megtett körök számát. Feladata, hogy minden körben megvizsgálja, hogy a robotok elérték-e a következő checkpointot.

\subsection{MapBuilder}
Fájlból beolvassa és létrehozza a memóriában a pályát, a kezdő pozíciókat és a checkpointokat reprezentáló objektumokat.  Mivel a  MapBuilder objektum tárolja a pályát így feladat, hogy vizsgálja a robotok, akadályok azon belül tartózkodását.  

\subsection{Robot}
Olyan objektum, mely a pályán található robotokat valósítja meg. Leírja a viselkedésüket és a kezelésüket. A „Robot” osztály a Unit-ból származik le, ezáltal van pozíciója és az ütközés is le van kezelve. Felelős a mozgásért, megállapítja egy adott akadállyal vagy robottal ütközött-e és kezeli a robot által felhasználható akadálykészleteket, illetve tartalmaz gombnyomást lekezelő metódusokat is.

\subsection{MyListener}
\comment{...}

\section{Osztályok leírása}

\subsection{Glue}
\begin{itemize}
\item Felelősség\\
A játékban szereplő Ragacs foltok viselkedését leíró osztály
\item Ősosztályok\\
Unit$\rightarrow$Obstacle
\item Metódusok
	\begin{itemize}
	    \item \textbf{Glue}(int x, int y): Amikor létrehozunk valahol egy ragacs objektumot, akkor ez a konstruktor hívódik meg. Meghívja az ősosztály konstruktorát.
		\item void \textbf{effect}(Robot r): Ütközéskor hívja meg az ütközést vizsgáló függvénye a Robot osztálynak. Módosítja a robot slowed értékét 50\%-ra a robot slowed attribútum setterének meghívásával.
	\end{itemize}
\end{itemize}

\subsection{GUI}
\begin{itemize}
\item Felelősség\\
A grafikus felületért felelős osztály, amely a menüt és a játékot jeleníti meg.
\item Attribútumok
	\begin{itemize}
		\item \textbf{Phoebe} game: referencia a játékra
	\end{itemize}
\item Metódusok
	\begin{itemize}
		\item\textbf{GUI}(): Konstruktor. Beállítja az ablak nevét, létrehozza az ablak elemeit, elrendezi őket és beállítja a figyelőket(ActionListener).
	\end{itemize}
\end{itemize}

\subsection{HUD}
\begin{itemize}
\item Felelősség\\
A robotok megtett köreit és checkpontjait tartja számon. Megvalósítja a checkpoint ellenőrzést.
\item Attribútumok
	\begin{itemize}
		\item \textbf{int[]} checkpointReached: Minden robothoz külön tárolja a legutoljára érintett checkpoint sorszámát.
		\item \textbf{int[]} lap: Minden robothoz tárolja a megtett körök számát. 
		\item \textbf{List} checkpoints: Tárolja a checkpointokat reprezentáló objektumokat List adatszerkezetben. A checkpointSearch függvény kérdezi le ebből a következő checkpoint helyzetét. 
		\item \textbf{List<Robot>} robots: A robotokat tároló List adatszerkezet. A checkpointSearch függvény kérdezi le ebből a robotokat, majd azok helyzetét.
	\end{itemize}
\item Metódusok
	\begin{itemize}
		\item \textbf{HUD}(List<Robot> robs): Robot objektumokat tároló ArrayList. Célja, hogy a checkpointsearch() függvényben minden robotra elvégezzük a keresést.
		\item void \textbf{checkpointSearch}(): Minden híváskor ellenőrzi, hogy a robot és a checkpoint metszete üres-e. 
		\item int \textbf{endOfTheGame}(): A játék végén eldönti, hogy melyik játékos nyert. Visszatér egy számmal, amiből egyértelműen eldönthető, hogy ki nyert. Ha negatív akkor az 1-es számú játékos nyert, ha nulla akkor döntetlen, ha pozitív akkor a 2-es számú játékos nyert.
		\item void \textbf{setCheckpointReached}(Robot r): Ha a paraméterként átadott robot következő checkpointja a célvonal (utolsó checkpoint) akkor lenullázza a checkpointReached-et és növeli a megtett körök számát, illetve ha nem akkor növeli az érintett checkpointok számát.
		\item void \textbf{setCheckpoints}(List checkObj): Checkpointokat reprezentáló adatszerkezet betöltése.CheckpointReached inicializálása a checkpointok számától függően.
	\end{itemize}
\end{itemize}

\subsection{IVisible}
\begin{itemize}
\item Felelősség\\
A grafikus motorhoz szükséges interfész. Olyan osztályok, melyek kirajzolható elemeket tartalmaznak megvalósítják ezt az interfészt.
\item Metódusok
	\begin{itemize}
		\item void \textbf{paint}(Graphics2D g): Rajzolást elvégző metódus.
	\end{itemize}
\end{itemize}

\subsection{MapBuilder}
\begin{itemize}
\item Felelősség\\
A pálya felépítéséért, a checkpointok tárolásáért és a robot pályán tartózkodásának vizsgálatáért felelős osztály.
\item Attribútumok
	\begin{itemize}
		\item \textbf{List} checkpoints: Tárolja a checkpointokat reprezentáló objektumokat List adatszerkezetben.
		\item \textbf{Object} map: A pályát reprezentáló objektum.
		\item \textbf{int[]} startPosPlayerOne: Meghatároz egy (x,y) koordinátát, ahol az első játékos kezd.
		\item \textbf{int[]} startPosPlayerTwo: Meghatároz egy (x,y) koordinátát, ahol az második játékos kezd.
	\end{itemize}
\item Metódusok
	\begin{itemize}
		\item \textbf{MapBuilder}(): Konstruktor, a pálya beolvasása fájlból, majd létrehozása.
		\item int[] \textbf{getStartPosPlayer}(int id): Paraméterül kap egy Robot id-t, majd visszatér egy int tömbbel, melyben található a robot kezdőpozíciója a pályán.
		 \item boolean \textbf{obstacleOutsideOfMap}(Obstacle obs): Egy akadályt vizsgál, hogy a pályán van-e.
		\item boolean \textbf{robotOutsideOfMap}(Robot r): Igaz értéket ad vissza, ha a robot leesett a pályáról, hamisat ha még rajta van.
	\end{itemize}
\end{itemize}

\subsection{MyListener}
\begin{itemize}
\item Felelősség\\
Nyilvántartja a gombok lenyomását és felengedését. Meghívja a gombhoz tartozó robotnak a gombnyomást lekezelő metódusát.
\item Interfészek\\
KeyListener, Runnable
\item Attribútumok
	\begin{itemize}
	    \item \textbf{List} robots: \comment{...}
		\item \textbf{boolean} isUp: \comment{...}
		\item \textbf{boolean} isDown: \comment{...}
		\item \textbf{boolean} isRight: \comment{...}
		\item \textbf{boolean} isLeft: \comment{...}
		\item \textbf{boolean} isW: \comment{...}
		\item \textbf{boolean} isD: \comment{...}
		\item \textbf{boolean} isS: \comment{...}
		\item \textbf{boolean} isA: \comment{...}
	\end{itemize}
\item Metódusok
	\begin{itemize}
		\item \textbf{MyListener}(List robots): \comment{...}
		\item void \textbf{run}():  \comment{...}
		\item void \textbf{keyPressed}(KeyEvent e): \comment{...}
		\item void \textbf{keyTyped}(KeyEvent e): \comment{...}
	\end{itemize}
\end{itemize}

\subsection{Obstacle}
\begin{itemize}
\item Felelősség\\
A pályán/játékosoknál lévő különböző akadályokat (ragacs,olaj) összefogó ősosztály.
\item Ősosztályok\\
Unit
\item Attribútumok
	\begin{itemize}
		\item \textbf{int} lifetime: Megmondja, hogy hány kör óta lett letéve az akadály.
	\end{itemize}
		\item \textbf{int} HEIGHT: Az akadályokat jellemző hosszúság. Szükség van rá, hogy létrehozzuk a leszármazottak hitbox-át(sokszög pályaelem).
		\item \textbf{int} WIDTH: Az akadályokat jellemző szélesség. Szükség van rá, hogy létrehozzuk a leszármazottak hitbox-át(sokszög pályaelem).
		
\item Metódusok
	\begin{itemize}
		\item \textbf{Obstacle}(int x, int y, String imagelocation): Meghívja a Unit konstruktorát a megadott adatokkal és létrehoz egy sokszög elemet ami reprezentálja a pályán majd.
		\item void \textbf{effect}(Robot r): Meghatározza, milyen hatással van a robotra, ha érintkezik egy Obstacle-lel. Absztrakt metódus.
		\item void \textbf{move}(): Mozgatásért felelős függvény. Ősosztályból öröklött, felülírt függvény. Mivel nem lehetséges az akadályok mozgása, ezért üres a függvény törzse.
	\end{itemize}
\end{itemize}

\subsection{Oil}
\begin{itemize}
\item Felelősség\\
A pályára lerakható olaj megvalósítása. Ha belelép egy játékos egy ilyen olajfoltba, az effect függvény letiltja a mozgatást az adott roboton a következő ugrásig.
\item Ősosztályok\\
Unit $\rightarrow$ Obstacle 
\item Metódusok
	\begin{itemize}
		\item \textbf{Oil}(int x, int y, String imagelocation): Egy Oil elem létrehozásáért felelős.
		\item void \textbf{effect}(Robot r): Meghatározza, milyen hatással van a robotra, ha beleugrik egy olajfoltba. Ebben az esetben letiltja a játékost, hogy irányt váltson.
	\end{itemize}
\end{itemize}

\subsection{Phoebe}
\begin{itemize}
\item Felelősség\\
A játék motorját képviselő osztály. A robotok pozíciójáért, az akadályok elhelyezéséért és a játék végéért felel.
\item Attribútumok
	\begin{itemize}
		\item \textbf{boolean} ended: Állapot változó, ha vége a játéknak, akkor true. Ha beteljesül egy játék végét jelentő esemény, akkor ezen a változón keresztül leáll a játék és megállapítódik a nyertes.
		\item \textbf{Settings} gameInfo: A játék beállításait tartalmazó osztály referenciája.
		\item \textbf{MyTimer} gameTimer: A játékhoz tartozó számlálóra mutató referencia.
		\item \textbf{HUD} hud: A játékosok előrehaladását, ragacs és olajkészleteit tartja számon.
		\item \textbf{MapBuilder} map: A pályát reprezentáló objektum. Tárolja még a checkpointokat és a robotok kezdő koordinátáit.
		\item \textbf{List<Obstacle>} obstacles: A játékban szereplő akadályok listája.
		\item \textbf{List<Robot>} robots: A játékban szereplő robotok listája.
	\end{itemize}
\item Metódusok
	\begin{itemize}
		\item \textbf{Phoebe}(Setting set): A játék felépítése, a robotok lista, az akadályok lista létrehozása.
		\item void \textbf{addObstacle}(Obstacle ob): Az obstacles tárolóba helyez egy akadályt.
		\item void \textbf{init}(): Inicializáló függvény. Feladat, hogy robotokat, akadályokat készítsen a játék kezdete előtt, létrehozza a pályát.
		\item void \textbf{paint}(Graphics2D g): Rajzoló függvény.
		\item void \textbf{run}(): Ez a metódus futtatja a főciklust, amelyben maga a játék működik.
	\end{itemize}
\end{itemize}

\subsection{Robot}
\begin{itemize}
\item Felelősség\\
A játékban résztvevő robotok viselkedését és kezelését leíró osztály.
\item Interfészek\\
IVisible, Unitból származva
\item Ősosztályok\\
Unit
\item Attribútumok
	\begin{itemize}
	    \item \textbf{int} arrowendx: A robot irányítását segítő nyílnak az x koordinátája, a nyíl kirajzolásánál van szerepe.
		\item \textbf{int} arrowendy: A robot irányítását segítő nyílnak az y koordinátája, a nyíl kirajzolásánál van szerepe.
		\item \textbf{double} alpha: A robot irányítását segítő nyíl vízszintessel bezárt szöge. A nyil kirajzolásánál, az ugrás végpontjának meghatározánál van szerepe.
		\item \textbf{int} HEIGHT: A robot képének magassága, collision                      detektálásnál, továbbá az irányítást segítő nyíl kezdő koordinátájának                  meghatározásánál szükséges.
		\item \textbf{int} WIDTH: A robot képének szélessége, funkcionalitásban hasonló a WIDTH-hez.
    	\item \textbf{int} ID: A robot példányának egyedi azonosítója, a keyconfig sorának indexelésére szükséges.
    	\item \textbf{boolean} leftobstacle: Logikai változó, tárolja, hogy az adott körben raktunk-e le már akadályt.
    	\item \textbf{boolean} moved: Azt jelöli, hogy lépett-e már a robot az aktuális körben. A megjelenítésnél (a nyilat ugrás közben nem jelenítjük meg), illetve az irányítás letiltásánál van szerepe (olajba lépés esetén).
		\item \textbf{int} numGlue: A robot számára felhasználható ragacsok száma.
	    \item \textbf{int} numOil: A robot számára felhasználható olajok száma.
	    \item \textbf{boolean} oiled: Azt jelzi, hogy olajba lépett-e. Ennek hatására a mozgás iránya módosíthatatlanná válik egy kis időre.
    	\item \textbf{Phoebe} p: Referencia a játékmotorra.
    	\item \textbf{int} r: A robot következő ugrását mutató nyíl hossza
    	\item \textbf{double} slowed: A sebesség módosításáért felel, default értéke 1.0. Amennyiben ragacsba lép a robot ez 0.5-re módosul és minden ugrás végén visszaáll az eredeti értékére. Ugrásnál ezzel szorozzuk be a végkordinátát kiszámító sugár hosszát.
		\item \textbf{int} staticID: Az osztályhoz tartozó statikus azonosító, a példány azonosítójának(id) meghatározásához szükséges.
\end{itemize}
\item Metódusok

	\begin{itemize}
		\item \textbf{Robot}(int x,int y,String imagelocation,Phoebe p): Létrehoz egy robotot a megadott x,y koordinátákon, betölti a képét az imagelocation cím alapján, eltárolja a játékmotor referenciáját, továbbá inicializálja felhasználható akadályok számát.
		\item void \textbf{bounce}(): Két robot ütközése után meghívódó függvény. Az ütközés függvényében kiszámolja a lepattanás irányát és letiltja egy körre az irány változtatást.
		\item boolean \textbf{collisionWithObstacle}(Obstacle o): Ellenőrzi hogy a robot ütközött-e az akadállyal. Igazzal tér vissza ha igen, hamissal ha nem.
		\item void \textbf{collisionWithRobot}(Robot r): Ellenőrzi, hogy a robot ütközött-e másik robottal. Ha igen, akkor gondoskodik róla, hogy a robotok a megfelelő szögben pattanjanak le egymásról.
		\item void \textbf{deathanimation}(): A Robot halálának grafikus megjelenítéséért felelős függvény.
		\item int \textbf{getNumGlue}(): Visszatér a felhasználható ragcsok számával.
		\item int \textbf{getNumOil}(): Visszatér a felhasználható olajok számával.
		\item void \textbf{incNumGlue}(): Növeli a robotnál tárolt ragacsok számát.
		\item void \textbf{incNumOil}(): Növeli a robotnál tárolt olajok számát.
		\item void \textbf{keyPressed}(int e): A robot irányítását megvalósító függvény, a játékmotor keylistener-e által hívódik meg, a lenyomott billentyű keyevent-jére. A következő ugrás beállítása, a ragacs/olaj lerakása történhet itt.
		\item void \textbf{move}(): A robot mozgását megvalósító függvény.
		\item void \textbf{paint}(Graphics2D g): Rajzoló függvény.
		\item void \textbf{setOiled}(): Átállítja a robot olaj effekt követéséhez tartozó állapot változót.
		\item void \textbf{setGlue}(): Átállítja a robot ragacs effekt követéséhez tartozó állapot változót.

	\end{itemize}
\end{itemize}

\subsection{MyTimer}
\begin{itemize}
\item Felelősség\\
A játék elején a kezdésig visszaszámol. Játéktípustól függően felfelé vagy visszafelé számol. Ez az osztály felelős, hogy ha lejárt az idő, akkor legyen vége a játéknak.
\item Metódusok
	\begin{itemize}
		\item \textbf{MyTimer}(int i): Konstruktor; inicializál egy viszaszámláló órát.
		\item boolean \textbf{isZero}(): Igazzal tér vissza, ha a megadott idő lejárt.
		\item int \textbf{getTime}(): Visszaadja az eltelt időt másodpercben.
		\item void \textbf{start}(): Elindítja / újraindítja a számlálót.
	\end{itemize}
\end{itemize}

\subsection{Unit}
\begin{itemize}
\item Felelősség\\
A pályán található objektumokért felel és azok viszonyáról (például ütközésükről).
\item Attribútumok
	\begin{itemize}
		\item \textbf{Object} hitbox: Az egységet a pályán reprezentáló sokszög.
		\item \textbf{int} x: Az egység x koordinátája
		\item \textbf{int} y: Az egység y koordinátája
	\end{itemize}
\item Metódusok
	\begin{itemize}
	    \item \textbf{Unit}(): A Unit osztály konstruktora. Feladata, hogy eltárolja az x,y koordinátát.
		\item boolean \textbf{intersect}(Unit u): Két egység ütközését meghatározó függvény.
		\item void \textbf{move}() : Absztrakt függvény, mely a leszármazottakban fog megvalósulni. Az egységek mozgásáért felelős.
	\end{itemize}
\end{itemize}

\setcounter{chapter}{6}
\chapter{Prototípus koncepciója}

\thispagestyle{fancy}

\section{Prototípus interface-definíciója}
\comment{Definiálni kell a teszteket leíró nyelvet. Külön figyelmet kell fordítani arra, hogy ha a rendszer véletlen elemeket is tartalmaz, akkor a véletlenszerűség ki-bekapcsolható legyen, és a program determinisztikusan is tesztelhető legyen.}

\subsection{Az interfész általános leírása}
\comment{A protó (karakteres) input és output felületeit úgy kell kialakítani, hogy az input fájlból is vehető legyen illetőleg az output fájlba menthető legyen, vagyis kommunikációra csak a szabványos be- és kimenet használható.}

\subsection{Bemeneti nyelv}
\comment{Definiálni kell a teszteket leíró nyelvet. Külön figyelmet kell fordítani arra, hogy ha a rendszer véletlen elemeket is tartalmaz, akkor a véletlenszerűség ki-bekapcsolható legyen, és a program determinisztikusan is futtatható legyen. A szálkezelést is tesztelhető, irányítható módon kell megoldani.}

\begin{itemize}
\item Parancs1
	\begin{itemize}
	\item Leírás:
	\item Opciók:
	\end{itemize}
\item Parancs2
	\begin{itemize}
	\item Leírás:
	\item Opciók:
	\end{itemize}

\end{itemize}

\comment{Ha szükséges, meg kell adni a konfigurációs (pl. pályaképet megadó) fájlok nyelvtanát is.}

\subsection{Kimeneti nyelv}
\comment{Egyértelműen definiálni kell, hogy az egyes bemeneti parancsok végrehajtása után előálló állapot milyen formában jelenik meg a szabványos kimeneten.}
jelölés:<osztály tagváltozója>
\begin{itemize}
\item Robot: 
	\begin{itemize}
	\item ToString():(nem írja ki automatikusan a szöveget csak vissza ad egy stringet, amit a run be kell kiíratni)
	        \begin{itemize}
	\item "Robot [id=<id>,  slowed=<slowed>,oiled=<oiled>, x=<x>,y=<y>,numGlue=<numGlue>,numOil=<numOil>,nextx=<arrowendx>,nexty=
	        <arrowendy>,alpha=<alpha>,width=<WIDTH>,height=<HEIGHT>]" 
	        \end{itemize}
	\item Keypressed(int k):
	       \begin{itemize}
	        \item "nextx ,nexty modified to:<arrowendx>,<arrowendy>"
            \item "new oil created at: <x>,<y>" ( ha volt olajunk és k == VK\_DOWN)
            \item "not enough oil"( ha nincs olajunk és k==VK\_DOWN)
            \item "new glue created at:<x>,<y>"(ha volt ragacsunk és k== VK\_UP)
            \item "not enough glue"( ha nincs ragacsunk és k==VK\_UP)
        
	       \end{itemize}
	 \item CollisionWithRobot(Robot robot2):
	       	
	        \begin{itemize}
	        \item  "there was a collision between this: <this.toString()> and this: <robot2.toString()>"
	        
	        \end{itemize}
    \item deathAnimation():
	        \begin{itemize}
	        \item  "This:<robot.toString()> died"
	        
	        \end{itemize}
	 \item CollisionWithObstacle(Obstacle o):
	       	
	        \begin{itemize}
	        \item  "there was a collision between this: <this.toString()> and this: <o.toString()>"
	        
	        \end{itemize}
	\end{itemize}
	
	
\item Oil:
	\begin{itemize}
	\item effect(robot r):
	        \begin{itemize}
	        \item  "you jumped into oil"
	        \end{itemize}
	\item toString():
	       \begin{itemize}
	        \item "Oil[x=<x>, y=<y>, Width=<WIDTH>, Height=<HEIGHT>, remainingturns=<Turnsremaining>]"
	       \end{itemize}
	\end{itemize}
\item Glue:
	\begin{itemize}
	\item effect(robot r):
	        \begin{itemize}
	        \item "you have been glued"
	        \end{itemize}
	\item toString():
	       \begin{itemize}
	        \item"Glue [x=<x>, y=<y>, Width=<WIDTH>, Height=<HEIGHT>,remaininglife=<Liferemaining>]";
	       \end{itemize}
	\end{itemize}

\item Cleaner:
	\begin{itemize}
	\item toString():
	       \begin{itemize}
	        \item "Cleaner[x=<x>, y=<y>, Width=<WIDTH>, Height=<HEIGHT>, State=<state>]"
	       \end{itemize}
	 \item CollisionWithRobot(Robot robot2):     
	        \begin{itemize}
	        \item  "there was a collision between this: <this.toString()> and this: <robot2.toString()>"
	        
	        \end{itemize}
	 \item cleanUp(Obstacle o):     
	        \begin{itemize}
	        \item  " this: <o.toString()> was cleaned up"
	        
	        \end{itemize}
\item CollisionWithObstacle(Obstacle o):     
	        \begin{itemize}
	        \item  "there was a collision between this: <this.toString()> and this: <o.toString()>"
	        
	        \end{itemize}
	        
	\end{itemize}
\end{itemize}

\section{Összes részletes use-case}
\comment{A use-case-eknek a részletezettsége feleljen meg a kezelői felületnek, azaz a felület elemeire kell hivatkozniuk.
Alábbi táblázat minden use-case-hez külön-külön.}

\begin{figure}[h]
\begin{center}
%\includegraphics[width=17cm]{chapters/chapter07/example.pdf}
\caption{x}
\label{fig:ProtoUseCase}
\end{center}
\end{figure}

\usecase{listObstacles}
{Kilistázza az akadályokat.}
{Tesztelő}
{Kiírja a kimenetre a játékban levő akadályokat (ragacsokat és olajfoltokat).}

\usecase{chechpointSearch}
{A robot egy checkpointba ugrott.}
{Tesztelő}
{A program megnézi, hogy a robot olyan területre ugrott, ahol checkpoint található. Ha beleugrott, megnöveli a robot „checkpoint elérésének” számát.}

\usecase{effect}
{Akadály hatása a robotra.}
{Tesztelő}
{Ha robot olajfoltra lép, akkor letiltja az irányváltást, így amilyen irányba ráugrott, olyan irányba tud továbbugrani.}

\usecase{move}
{Robot irányának beállítása.}
{Tesztelő}
{Robot ugrása előtt a tesztelő beállítja a robot irányát az irányváltoztató gombokkal (0-180).}

\usecase{listCleaners}
{Kilistázza a kisrobotokat.}
{Tesztelő}
{Kiírja a kimenetre a játékban levő kisrobotokat.}

\usecase{collisionWithObstacles}
{Azonos koordinátákon van robot és az akadály.}
{Tesztelő}
{Ha a robot azonos koordinátákon van egy olajfolttal, akkor beállítja a robotot a kívánt hatásra.}

\usecase{keyPressed}
{A robot koordinátáira akadályt tesz le.}
{Tesztelő}
{Robot ugrása előtt a tesztelő az adott gombbal akadályt rak le a robot helyére.}

\usecase{listRobots}
{Kilistázza a robotokat}
{Tesztelő}
{Kiírja a kimenetre a játékban levő robotokat.}

\usecase{collisionWithCleaner}
{Robot és kisrobot azonos koordinátákon helyezkednek el a földön.}
{Tesztelő}
{Ha a robot és kisrobottal azonos koordinátákon helyezkedik el akkor a kisrobot meghal.}

\usecase{collisionWithRobot}
{Robot és robot azonos koordinátákon helyezkednek el.}
{Robot}
{Ha két robot azonos koordinátákon helyezkednek el és:  
- ha ugyanakkora a sebességük: lepattannak egymásról
- ha különböző: a lassabb meghal
}

\usecase{robotOutsideOfMap}
{Robot kiugrik a pályáról.}
{Tesztelő}
{Ha a robot kiugrott a pályáról meghalt.}

\usecase{setCheckpoints}
{Checkpointok beállítása}
{Tesztelő}
{Beállítja a pályára a checkpointokat.}

\usecase{MapBuilder}
{Pálya betöltése}
{Tesztelő}
{A tesztelő által kiválasztott pálya betöltése.}

\usecase{Glue}
{Ragacs létrehozása}
{Tesztelő}
{A tesztelő a pálya megadott koordinátáira ragacsot helyez el.}

\usecase{Oil}
{Olajfolt létrehozása}
{Tesztelő}
{A tesztelő a pálya megadott koordinátáira olajfoltot helyez el.}




\section{Tesztelési terv}
\comment{A tesztelési tervben definiálni kell, hogy a be- és kimeneti fájlok egybevetésével miként végezhető el a program tesztelése. Meg kell adni teszt forgatókönyveket. Az egyes teszteket elég informálisan, szabad szövegként leírni. Teszt-esetenként egy-öt mondatban. Minden teszthez meg kell adni, hogy mi a célja, a proto mely funkcionalitását, osztályait stb. teszteli. Az alábbi táblázat minden teszt-esethez külön-külön elkészítendő.}

\teszteset{Mozgás}	{A robot mozgásának tesztelése.}	{Teszteli, hogy a robot mozgása után az előre megadott értékeknek megfelelően történik a mozgás (a mozgás után a megfelelő helyen van).}
\teszteset{Irányváltoztatás}	{A robot irányváltoztatásának tesztelése.}	{Teszteli, hogy a robot különböző irányok beállítása után is a megfelelő helyre ugrik.}
\teszteset{Ütközés}	{A robotok ütközésének tesztelése.}	{Teszteli, hogy a két robot ütközése után a megfelelő helyre pattannak-e, illetve hogy az objektumok határai helyesen működnek.}
\teszteset{Leesés}	{A robot leesésének tesztelése.}	{Teszteli, hogy a pálya szélén a robot megszűnik-e létezni.}
\teszteset{Checkpoint}	{A robot checkpoint-ba érésének tesztelése.}	{Teszteli, hogy a checkpoint-ba érést, illetve hogy a robotnak megfelelően változnak-e az értékei ilyen esetben.}
\teszteset{Olajba ugrás}	{A robot olajba ugrását teszteli.}	{Teszteli, hogy egy olajba ugrás alkalmával változnak-e a robot értékei, vizsgálja az ütközést.}
\teszteset{Ragacsba ugrás}	{A robot ragacsba ugrását teszteli.}	{Teszteli, hogy egy ragacsba ugrás alkalmával változnak-e a robot értékei, vizsgálja az ütközést.}
\teszteset{Olaj lerakás}	{Olaj lerakása a pályán.}	{Teszteli, hogy a lerakás után létrejött-e az olaj a pályán.}
\teszteset{Ragacs lerakás}	{Ragacs lerakása a pályán.}	{Teszteli, hogy a lerakás után létrejött-e a ragacs a pályán.}
\teszteset{Olaj hatás}	{Az olaj hatását teszteli.}	{Teszteli, hogy olajos állapotban a robot képes-e irányt változtatni.}
\teszteset{Ragacs hatás}	{A ragacs hatását teszteli.}	{Teszteli, hogy ragacsos állapotban a robot milyen távolságra képes ugrani.}

\section{Tesztelést támogató segéd- és fordítóprogramok specifikálása}
\comment{Specifikálni kell a tesztelést támogató segédprogramokat.}


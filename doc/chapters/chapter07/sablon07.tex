%\textit{% Szglab4
% ===========================================================================
%
\chapter{Prototípus koncepciója}

\thispagestyle{fancy}

\section{Prototípus interface-definíciója}
\comment{Definiálni kell a teszteket leíró nyelvet. Külön figyelmet kell fordítani arra, hogy ha a rendszer véletlen elemeket is tartalmaz, akkor a véletlenszerűség ki-bekapcsolható legyen, és a program determinisztikusan is tesztelhető legyen.}

\subsection{Az interfész általános leírása}
\comment{A protó (karakteres) input és output felületeit úgy kell kialakítani, hogy az input fájlból is vehető legyen illetőleg az output fájlba menthető legyen, vagyis kommunikációra csak a szabványos be- és kimenet használható.}

\subsection{Bemeneti nyelv}
\comment{Definiálni kell a teszteket leíró nyelvet. Külön figyelmet kell fordítani arra, hogy ha a rendszer véletlen elemeket is tartalmaz, akkor a véletlenszerűség ki-bekapcsolható legyen, és a program determinisztikusan is futtatható legyen. A szálkezelést is tesztelhető, irányítható módon kell megoldani.}

\begin{itemize}
\item Parancs1
	\begin{itemize}
	\item Leírás:
	\item Opciók:
	\end{itemize}
\item Parancs2
	\begin{itemize}
	\item Leírás:
	\item Opciók:
	\end{itemize}

\end{itemize}

\comment{Ha szükséges, meg kell adni a konfigurációs (pl. pályaképet megadó) fájlok nyelvtanát is.}

\subsection{Kimeneti nyelv}
\comment{Egyértelműen definiálni kell, hogy az egyes bemeneti parancsok végrehajtása után előálló állapot milyen formában jelenik meg a szabványos kimeneten.}
jelölés:<osztály tagváltozója>
\begin{itemize}
\item Robot: 
	\begin{itemize}
	\item ToString():(nem írja ki automatikusan a szöveget csak vissza ad egy stringet, amit a run be kell kiíratni)
	        \begin{itemize}
	\item "Robot [id=<id>,  slowed=<slowed>,oiled=<oiled>, x=<x>,y=<y>,nextx=<arrowendx>,nexty=
	        <arrowendy>,alpha=<alpha>,width=<WIDTH>,height=<HEIGHT>]” 
	        \end{itemize}
	\item Keypressed(int k):
	       \begin{itemize}
	        \item nextx ,nexty modified to:<arrowendx>,<arrowendy>
            \item new oil created at: <x>,<y> ( ha volt olajunk és k == VK\_DOWN)
            \item „not enough oil”( ha nincs olajunk és k==VK\_DOWN)
            \item "new glue created at:<x>,<y>”(ha volt ragacsunk és k== VK\_UP)
            \item „not enough glue”( ha nincs ragacsunk és k==VK\_UP)

	       \end{itemize}
	\end{itemize}
	
	
\item Oil:
	\begin{itemize}
	\item effect(robot r):
	        \begin{itemize}
	        \item  „you jumped into oil”
	        \end{itemize}
	\item toString():
	       \begin{itemize}
	        \item "Oil[x=<x>, y=<y>, Width=<WIDTH>, Height=<HEIGHT>]"
	       \end{itemize}
	\end{itemize}
\item Glue:
	\begin{itemize}
	\item effect(robot r):
	        \begin{itemize}
	        \item „you have been glued”
	        \end{itemize}
	\item toString():
	       \begin{itemize}
	        \item"Glue [x=<x>, y=<y>, Width=<WIDTH>, Height=<HEIGHT>]";
	       \end{itemize}
	\end{itemize}
\end{itemize}


\section{Összes részletes use-case}
\comment{A use-case-eknek a részletezettsége feleljen meg a kezelői felületnek, azaz a felület elemeire kell hivatkozniuk.
Alábbi táblázat minden use-case-hez külön-külön.}

\begin{figure}[h]
\begin{center}
%\includegraphics[width=17cm]{chapters/chapter07/example.pdf}
\caption{x}
\label{fig:ProtoUseCase}
\end{center}
\end{figure}

\usecase{...}{...}{...}{...}

\section{Tesztelési terv}
\comment{A tesztelési tervben definiálni kell, hogy a be- és kimeneti fájlok egybevetésével miként végezhető el a program tesztelése. Meg kell adni teszt forgatókönyveket. Az egyes teszteket elég informálisan, szabad szövegként leírni. Teszt-esetenként egy-öt mondatban. Minden teszthez meg kell adni, hogy mi a célja, a proto mely funkcionalitását, osztályait stb. teszteli. Az alábbi táblázat minden teszt-esethez külön-külön elkészítendő.}

\teszteset{...}{...}{...}

\section{Tesztelést támogató segéd- és fordítóprogramok specifikálása}
\comment{Specifikálni kell a tesztelést támogató segédprogramokat.}

}
%\textit{% Szglab4
% ===========================================================================
%
\setcounter{chapter}{-1}
\chapter{Módosítások}
\comment{Feladatok: Objektum katalógus szövegeinek ellenőrzése, Osztályok leírásának pontjainak ellenőrzése, újonnan hozzáadott elemeinek felelősségének leírása}
\section{Objektum katalógus}
\subsection{HUD}
Ez az objektum követi és nyilvántartja, hogy a robotok hány checkpoint-on mentek át, illetve kiírja a képernyőre a hátramaradó időt és a megtett körök számát. Feladata, hogy minden körben megvizsgálja, hogy a robotok elérték-e a következő checkpointot.
\subsection{MyListener}
\comment{...}

\section{Osztályok leírása}
\subsection{HUD}
\begin{itemize}
\item Felelősség\\
A robotok megtett köreit és checkpontjait tartja számon. Megvalósítja a checkpoint ellenőrzést.
\item Attribútumok
	\begin{itemize}
		\item \textbf{int[]} checkpointReached: Minden robothoz külön tárolja a legutoljára érintett checkpoint sorszámát.
		\item \textbf{int[]} lap: Minden robothoz tárolja a megtett körök számát. 
		\item \textbf{List} checkpoints: Tárolja a checkpointokat reprezentáló objektumokat List adatszerkezetben. A checkpointSearch függvény kérdezi le ebből a következő checkpoint helyzetét. 
		\item \textbf{List<Robot>} robots: A robotokat tároló List adatszerkezet. A checkpointSearch függvény kérdezi le ebből a robotokat, majd azok helyzetét.
	\end{itemize}
\item Metódusok
	\begin{itemize}
		\item \textbf{HUD}(List<Robot> robs): Robot objektumokat tároló ArrayList. Célja, hogy a checkpointsearch() függvényben minden robotra elvégezzük a keresést.
		\item void \textbf{checkpointSearch}(): Minden híváskor ellenőrzi, hogy a robot és a checkpoint metszete üres-e. 
		\item void \textbf{setCheckpoints}(List checkObj): Checkpointokat reprezentáló adatszerkezet betöltése.CheckpointReached inicializálása a checkpointok számától függően.
		\item void \textbf{setCheckpointReached}(Robot r): Ha a paraméterként átadott robot következő checkpointja a célvonal (utolsó checkpoint) akkor lenullázza a checkpointReached-et és növeli a megtett körök számát, illetve ha nem akkor növeli az érintett checkpointok számát.
		\item int \textbf{endOfTheGame}(): A játék végén eldönti, hogy melyik játékos nyert. Visszatér egy számmal, amiből egyértelműen eldönthető, hogy ki nyert. Ha negatív akkor az 1-es számú játékos nyert, ha nulla akkor döntetlen, ha pozitív akkor a 2-es számú játékos nyert.
	\end{itemize}
\end{itemize}

\subsection{IVisible}
\begin{itemize}
\item Felelősség\\
A grafikus motorhoz szükséges interfész. Olyan osztályok, melyek kirajzolható elemeket tartalmaznak megvalósítják ezt az interfészt.
\item Metódusok\\
	\begin{itemize}
		\item void \textbf{paint}(Graphics2D g): Rajzolást elvégző metódus.
	\end{itemize}
\end{itemize}

\subsection{MapBuilder}
\begin{itemize}
\item Felelősség\\
A pálya felépítéséért, a checkpointok tárolásáért és a robot pályán tartózkodásának vizsgálatáért felelős osztály.
\item Attribútumok
	\begin{itemize}
		\item \textbf{List} checkpoints: Tárolja a checkpointokat reprezentáló objektumokat List adatszerkezetben.
		\item \textbf{int[]} startPosPlayerOne: Meghatároz egy (x,y) koordinátát, ahol az első játékos kezd.
		\item \textbf{int[]} startPosPlayerTwo: Meghatároz egy (x,y) koordinátát, ahol az második játékos kezd.
		\item \textbf{Object} map: \comment{...}
	\end{itemize}
\item Metódusok
	\begin{itemize}
		\item \textbf{MapBuilder}(): Konstruktor, a pálya beolvasása fájlból, majd létrehozása.
		\item boolean \textbf{robotOutsideOfMap}(Robot r): Igaz értéket ad vissza, ha a robot leesett a pályáról, hamisat ha még rajta van.
	    \item boolean \textbf{obstacleOutsideOfMap}(Obstacle obs): \comment{...}
	    \item int[] \textbf{getStartPosPlayer}(int id): \comment{...}
	\end{itemize}
\end{itemize}

\subsection{MyListener}
\begin{itemize}
\item Felelősség\\
Nyilvántartja a gombok lenyomását és felengedését. Meghívja a gombhoz tartozó robotnak a gombnyomást lekezelő metódusát.
\item Interfészek\\
KeyListener, Runnable
\item Attribútumok\\
	\begin{itemize}
	    \item \textbf{List} robots: \comment{...}
		\item \textbf{boolean} isUp: \comment{...}
		\item \textbf{boolean} isDown: \comment{...}
		\item \textbf{boolean} isRight: \comment{...}
		\item \textbf{boolean} isLeft: \comment{...}
		\item \textbf{boolean} isW: \comment{...}
		\item \textbf{boolean} isD: \comment{...}
		\item \textbf{boolean} isS: \comment{...}
		\item \textbf{boolean} isA: \comment{...}
	\end{itemize}
\item Metódusok\\
	\begin{itemize}
		\item \textbf{MyListener}(List robots): \comment{...}
		\item void \textbf{run}():  \comment{...}
		\item void \textbf{keyPressed}(KeyEvent e): \comment{...}
		\item void \textbf{keyTyped}(KeyEvent e): \comment{...}
	\end{itemize}
\end{itemize}

\subsection{Obstacle}
\begin{itemize}
\item Felelősség\\
A pályán/játékosoknál lévő különböző akadályokat (ragacs,olaj) összefogó ősosztály.
\item Ősosztályok\\
Unit
\item Attribútumok
	\begin{itemize}
		\item \textbf{int} WIDTH: Az akadályokat jellemző szélesség. Szükség van rá, hogy létrehozzuk a leszármazottak hitbox-át(sokszög pályaelem).
		\item \textbf{int} HEIGHT: Az akadályokat jellemző hosszúság. Szükség van rá, hogy létrehozzuk a leszármazottak hitbox-át(sokszög pályaelem).
		\item \textbf{int} lifetime: Megmondja, hogy hány kör óta lett letéve az akadály.
	\end{itemize}
\item Metódusok
	\begin{itemize}
		\item \textbf{Obstacle}(int x, int y, String imagelocation): Meghívja a Unit konstruktorát a megadott adatokkal és létrehoz egy sokszög elemet ami reprezentálja a pályán majd.
		\item void \textbf{effect}(Robot r): Meghatározza, milyen hatással van a robotra, ha érintkezik egy Obstacle-lel. Absztrakt metódus.
		\item void \textbf{move}(): Mozgatásért felelős függvény. Ősosztályból öröklött, felülírt függvény. Mivel nem lehetséges az akadályok mozgása, ezért üres a függvény törzse.
	\end{itemize}
\end{itemize}

\subsection{Oil}
\begin{itemize}
\item Felelősség\\
A pályára lerakható olaj megvalósítása. Ha belelép egy játékos egy ilyen olajfoltba, az effect függvény letiltja a mozgatást az adott roboton a következő ugrásig.
\item Ősosztályok\\
Unit $\rightarrow$ Obstacle 
\item Metódusok
	\begin{itemize}
		\item \textbf{Oil}(int x, int y, String imagelocation): Egy Oil elem létrehozásáért felelős.
		\item void \textbf{effect}(Robot r): Meghatározza, milyen hatással van a robotra, ha beleugrik egy olajfoltba. Ebben az esetben letiltja a játékost, hogy irányt váltson.
	\end{itemize}
\end{itemize}

\subsection{Phoebe}
\begin{itemize}
\item Felelősség\\
A játék motorját képviselő osztály. A robotok pozíciójáért, az akadályok elhelyezéséért és a játék végéért felel.
\item Attribútumok
	\begin{itemize}
		\item \textbf{boolean} ended: Állapot változó, ha vége a játéknak, akkor true. Ha beteljesül egy játék végét jelentő esemény, akkor ezen a változón keresztül leáll a játék és megállapítódik a nyertes.
		\item \textbf{Settings} gameInfo: \comment{...}
		\item \textbf{List<Robot>} robots: A játékban szereplő robotok listája.
		\item \textbf{List<Obstacle>} obstacles: A játékban szereplő akadályok listája.
		\item \textbf{HUD} hud: A játékosok előrehaladását, ragacs és olajkészleteit tartja számon
		\item \textbf{MapBuilder} map: A pályát reprezentáló objektum. Tárolja még a checkpointokat és a robotok kezdő koordinátáit.
		\item \textbf{MyTimer} gameTimer: \comment{...}
	\end{itemize}
\item Metódusok
	\begin{itemize}
		\item \textbf{Phoebe}(Setting set): A játék felépítése, a robotok lista, az akadályok lista létrehozása.
		\item void \textbf{addObstacle}(Obstacle ob): Az obstacles tárolóba helyez egy akadályt.
		\item void \textbf{run}(): Ez a metódus futtatja a főciklust, amelyben maga a játék működik.
		\item void \textbf{paint}(Graphics2D g): \comment{...}
		\item void \textbf{init}(): \comment{...}
	\end{itemize}
\end{itemize}

\subsection{Robot}
\begin{itemize}
\item Felelősség\\
A játékban résztvevő robotok viselkedését és kezelését leíró osztály.
\item Interfészek\\
IVisible, Unitból származva
\item Ősosztályok\\
Unit
\item Attribútumok
	\begin{itemize}
    	\item \textbf{Phoebe} p: Referencia a játékmotorra.
		\item \textbf{int} staticID: Az osztályhoz tartozó statikus azonosító, a példány                azonosítójának(id) meghatározásához szükséges.
		\item \textbf{int} ID: A robot példányának egyedi azonosítója, a keyconfig sorának indexelésére szükséges.
		\item \textbf{int} HEIGHT: A robot képének magassága, collision                      detektálásnál, továbbá az irányítást segítő nyíl kezdő koordinátájának                  meghatározásánál szükséges.
		\item \textbf{int} WIDTH: A robot képének szélessége, funkcionalitásban hasonló a WIDTH-hez.

		\item \textbf{double} slowed: A sebesség módosításáért felel, default értéke 1.0. Amennyiben ragacsba lép a robot ez 0.5-re módosul és minden ugrás végén visszaáll az eredeti értékére. Ugrásnál ezzel szorozzuk be a végkordinátát kiszámító sugár hosszát.
		\item \textbf{boolean} oiled: Azt jelzi, hogy olajba lépett-e. Ennek hatására a mozgás iránya módosíthatatlanná válik egy kis időre.
		\item \textbf{int} r: \comment{...}
		\item \textbf{int} arrowendx: A robot irányítását segítő nyílnak az x koordinátája, a nyíl kirajzolásánál van szerepe.
		\item \textbf{int} arrowendy: A robot irányítását segítő nyílnak az y koordinátája, a nyíl kirajzolásánál van szerepe.
		\item \textbf{double} alpha: A robot irányítását segítő nyíl vízszintessel bezárt szöge. A nyil kirajzolásánál, az ugrás végpontjának meghatározánál van szerepe.
		\item \textbf{boolean} moved: Azt jelöli, hogy lépett-e már a robot az aktuális körben. A megjelenítésnél (a nyilat ugrás közben nem jelenítjük meg), illetve az irányítás letiltásánál van szerepe (olajba lépés esetén).
		\item \textbf{int} numGlue: \comment{...}
	    \item \textbf{int} numOil: \comment{...}
	    \item \textbf{boolean} leftobstacle: \comment{...}
\end{itemize}
\item Metódusok

	\begin{itemize}
		\item \textbf{Robot}(int x,int y,String imagelocation,Phoebe p): Létrehoz egy robotot a megadott x,y koordinátákon, betölti a képét az imagelocation cím alapján, eltárolja a játékmotor referenciáját, továbbá inicializálja felhasználható akadályok számát.
		\item void \textbf{setOiled}(): Átállítja a robot olaj effekt követéséhez tartozó állapot változót.
		\item void \textbf{setGlue}(): Átállítja a robot ragacs effekt követéséhez tartozó állapot változót.
		\item int \textbf{getNumGlue}(): Visszatér a felhasználható ragcsok számával.
		\item int \textbf{getNumOil}(): Visszatér a felhasználható olajok számával.
		\item void \textbf{incNumGlue}(): Növeli a robotnál tárolt ragacsok számát.
		\item void \textbf{incNumOil}(): Növeli a robotnál tárolt olajok számát.
		\item \textbf{void deathanimation}(): A Robot halálának grafikus megjelenítéséért felelős függvény.
		\item boolean \textbf{collisionWithObstacle}(Obstacle o): Ellenőrzi hogy a robot ütközött-e az akadállyal. Igazzal tér vissza ha igen, hamissal ha nem.
		\item void \textbf{collisionWithRobot}(Robot r): Ellenőrzi, hogy a robot ütközött-e másik robottal. Ha igen, akkor gondoskodik róla, hogy a robotok a megfelelő szögben pattanjanak le egymásról.
		\item void \textbf{bounce}(): Két robot ütközése után meghívódó függvény. Az ütközés függvényében kiszámolja a lepattanás irányát és letiltja egy körre az irány változtatást.
		\item void \textbf{move}(): A robot mozgását megvalósító függvény.
		\item void \textbf{keyPressed}(int e): A robot irányítását megvalósító függvény, a játékmotor keylistener-e által hívódik meg, a lenyomott billentyű keyevent-jére. A következő ugrás beállítása, a ragacs/olaj lerakása történhet itt.
		\item void \textbf{paint}(Graphics2D g): \comment{...}
	\end{itemize}
\end{itemize}

\subsection{MyTimer}
\begin{itemize}
\item Felelősség\\
A játék elején a kezdésig visszaszámol. Játéktípustól függően felfelé vagy visszafelé számol. Ez az osztály felelős, hogy ha lejárt az idő, akkor legyen vége a játéknak.
\item Metódusok
	\begin{itemize}
		\item \textbf{MyTimer}(int i): Konstruktor; inicializál egy viszaszámláló órát.
		\item boolean \textbf{isZero}(): Igazzal tér vissza, ha a megadott idő lejárt.
		\item void \textbf{start}(): \comment{...}
		\item int \textbf{getTime}(): \comment{...}
	\end{itemize}
\end{itemize}

\subsection{Unit}
\begin{itemize}
\item Felelősség\\
A pályán található objektumokért felel és azok viszonyáról (például ütközésükről).
\item Attribútumok
	\begin{itemize}
		\item \textbf{int} x: Az egység x koordinátája
		\item \textbf{int} y: Az egység y koordinátája
		\item \textbf{Object} hitbox: Az egységet a pályán reprezentáló sokszög.
	\end{itemize}
\item Metódusok
	\begin{itemize}
	    \item \textbf{Unit}(): A Unit osztály konstruktora. Feladata, hogy eltárolja az x,y koordinátát.
		\item void \textbf{move}() : Absztrakt függvény, mely a leszármazottakban fog megvalósulni. Az egységek mozgásáért felelős.
		\item boolean \textbf{intersect}(Unit u): Két egység ütközését meghatározó függvény.
	\end{itemize}
\end{itemize}

\setcounter{chapter}{6}
\chapter{Prototípus koncepciója}

\thispagestyle{fancy}

\section{Prototípus interface-definíciója}
\comment{Definiálni kell a teszteket leíró nyelvet. Külön figyelmet kell fordítani arra, hogy ha a rendszer véletlen elemeket is tartalmaz, akkor a véletlenszerűség ki-bekapcsolható legyen, és a program determinisztikusan is tesztelhető legyen.}

\subsection{Az interfész általános leírása}
\comment{A protó (karakteres) input és output felületeit úgy kell kialakítani, hogy az input fájlból is vehető legyen illetőleg az output fájlba menthető legyen, vagyis kommunikációra csak a szabványos be- és kimenet használható.}

\subsection{Bemeneti nyelv}
\comment{Definiálni kell a teszteket leíró nyelvet. Külön figyelmet kell fordítani arra, hogy ha a rendszer véletlen elemeket is tartalmaz, akkor a véletlenszerűség ki-bekapcsolható legyen, és a program determinisztikusan is futtatható legyen. A szálkezelést is tesztelhető, irányítható módon kell megoldani.}

\begin{itemize}
\item Parancs1
	\begin{itemize}
	\item Leírás:
	\item Opciók:
	\end{itemize}
\item Parancs2
	\begin{itemize}
	\item Leírás:
	\item Opciók:
	\end{itemize}

\end{itemize}

\comment{Ha szükséges, meg kell adni a konfigurációs (pl. pályaképet megadó) fájlok nyelvtanát is.}

\subsection{Kimeneti nyelv}
\comment{Egyértelműen definiálni kell, hogy az egyes bemeneti parancsok végrehajtása után előálló állapot milyen formában jelenik meg a szabványos kimeneten.}
jelölés:<osztály tagváltozója>
\begin{itemize}
\item Robot: 
	\begin{itemize}
	\item ToString():(nem írja ki automatikusan a szöveget csak vissza ad egy stringet, amit a run be kell kiíratni)
	        \begin{itemize}
	\item "Robot [id=<id>,  slowed=<slowed>,oiled=<oiled>, x=<x>,y=<y>,nextx=<arrowendx>,nexty=
	        <arrowendy>,alpha=<alpha>,width=<WIDTH>,height=<HEIGHT>]” 
	        \end{itemize}
	\item Keypressed(int k):
	       \begin{itemize}
	        \item nextx ,nexty modified to:<arrowendx>,<arrowendy>
            \item new oil created at: <x>,<y> ( ha volt olajunk és k == VK\_DOWN)
            \item „not enough oil”( ha nincs olajunk és k==VK\_DOWN)
            \item "new glue created at:<x>,<y>”(ha volt ragacsunk és k== VK\_UP)
            \item „not enough glue”( ha nincs ragacsunk és k==VK\_UP)

	       \end{itemize}
	\end{itemize}
	
	
\item Oil:
	\begin{itemize}
	\item effect(robot r):
	        \begin{itemize}
	        \item  „you jumped into oil”
	        \end{itemize}
	\item toString():
	       \begin{itemize}
	        \item "Oil[x=<x>, y=<y>, Width=<WIDTH>, Height=<HEIGHT>]"
	       \end{itemize}
	\end{itemize}
\item Glue:
	\begin{itemize}
	\item effect(robot r):
	        \begin{itemize}
	        \item „you have been glued”
	        \end{itemize}
	\item toString():
	       \begin{itemize}
	        \item"Glue [x=<x>, y=<y>, Width=<WIDTH>, Height=<HEIGHT>]";
	       \end{itemize}
	\end{itemize}
\end{itemize}


\section{Összes részletes use-case}
\comment{A use-case-eknek a részletezettsége feleljen meg a kezelői felületnek, azaz a felület elemeire kell hivatkozniuk.
Alábbi táblázat minden use-case-hez külön-külön.}

\begin{figure}[h]
\begin{center}
%\includegraphics[width=17cm]{chapters/chapter07/example.pdf}
\caption{x}
\label{fig:ProtoUseCase}
\end{center}
\end{figure}

\usecase{...}{...}{...}{...}

\section{Tesztelési terv}
\comment{A tesztelési tervben definiálni kell, hogy a be- és kimeneti fájlok egybevetésével miként végezhető el a program tesztelése. Meg kell adni teszt forgatókönyveket. Az egyes teszteket elég informálisan, szabad szövegként leírni. Teszt-esetenként egy-öt mondatban. Minden teszthez meg kell adni, hogy mi a célja, a proto mely funkcionalitását, osztályait stb. teszteli. Az alábbi táblázat minden teszt-esethez külön-külön elkészítendő.}

\teszteset{...}{...}{...}

\section{Tesztelést támogató segéd- és fordítóprogramok specifikálása}
\comment{Specifikálni kell a tesztelést támogató segédprogramokat.}


% Szglab4
% ===========================================================================
%
\thispagestyle{fancy}

\setcounter{chapter}{-1}
\chapter{Módosítások}
\section{Objektum katalógus}
\subsection{HUD}
Ez az objektum követi és nyilvántartja, hogy a robotok hány checkpoint-on mentek át, illetve kiírja a képernyőre a hátramaradó időt és a megtett körök számát. Feladata, hogy minden körben megvizsgálja, hogy a robotok elérték-e a következő checkpointot.
\subsection{MapBuilder}
Fájlból beolvassa és létrehozza a memóriában a pályát, a kezdő pozíciókat és a checkpointokat reprezentáló objektumokat.  Mivel a  MapBuilder objektum tárolja a pályát így feladat, hogy vizsgálja a robotok azon belül tartózkodását.  

\section{Osztályok leírása}

\subsection{Glue}
\begin{itemize}
\item Felelősség\\
A játékban szereplő Ragacs foltok viselkedését leíró osztály
\item Ősosztályok\\
Unit$\rightarrow$Obstacle
\item Metódusok
	\begin{itemize}
	    \item \textbf{Glue}(int x, int y): Amikor létrehozunk valahol egy ragacs objektumot, akkor ez a konstruktor hívódik meg. Meghívja az ősosztály konstruktorát.
		\item void \textbf{effect}(Robot r): Ütközéskor hívja meg az ütközést vizsgáló függvénye a Robot osztálynak. Módosítja a robot slowed értékét 50\%-ra a robot slowed attribútum setterének meghívásával.
	\end{itemize}
\end{itemize}

\subsection{GUI}
\begin{itemize}
\item Felelősség\\
A grafikus felületért felelős osztály, amely a menüt és a játékot jeleníti meg.
\item Attribútumok
	\begin{itemize}
		\item \textbf{Phoebe} game: referencia a játékra
	\end{itemize}
\item Metódusok
	\begin{itemize}
		\item\textbf{GUI}(): Konstruktor. Beállítja az ablak nevét, létrehozza az ablak elemeit, elrendezi őket és beállítja a figyelőket(ActionListener).
	\end{itemize}
\end{itemize}

\subsection{HUD}
\begin{itemize}
\item Felelősség\\
A robotok megtett köreit és checkpontjait tartja számon. Megvalósítja a checkpoint ellenőrzést.
\item Attribútumok
	\begin{itemize}
		\item \textbf{int[]} checkpointReached: Minden robothoz külön tárolja a legutoljára érintett checkpoint sorszámát.
		\item \textbf{int[]} lap: Minden robothoz tárolja a megtett körök számát. 
		\item \textbf{List} checkpoints: Tárolja a checkpointokat reprezentáló objektumokat List adatszerkezetben. A checkpointSearch függvény kérdezi le ebből a következő checkpoint helyzetét. 
		\item \textbf{List<Robot>} robots: A robotokat tároló List adatszerkezet. A checkpointSearch függvény kérdezi le ebből a robotokat, majd azok helyzetét.
	\end{itemize}
\item Metódusok
	\begin{itemize}
		\item \textbf{HUD}(List<Robot> robs): Robot objektumokat tároló ArrayList. Célja, hogy a checkpointsearch() függvényben minden robotra elvégezzük a keresést.
		\item void \textbf{checkpointSearch}(): Minden híváskor ellenőrzi, hogy a robot és a checkpoint metszete üres-e. 
		\item int \textbf{endOfTheGame}(): A játék végén eldönti, hogy melyik játékos nyert. Visszatér egy számmal, amiből egyértelműen eldönthető, hogy ki nyert. Ha negatív akkor az 1-es számú játékos nyert, ha nulla akkor döntetlen, ha pozitív akkor a 2-es számú játékos nyert.
		\item void \textbf{setCheckpointReached}(Robot r): Ha a paraméterként átadott robot következő checkpointja a célvonal (utolsó checkpoint) akkor lenullázza a checkpointReached-et és növeli a megtett körök számát, illetve ha nem akkor növeli az érintett checkpointok számát.
		\item void \textbf{setCheckpoints}(List checkObj): Checkpointokat reprezentáló adatszerkezet betöltése.CheckpointReached inicializálása a checkpointok számától függően.
	\end{itemize}
\end{itemize}

\subsection{IVisible}
\begin{itemize}
\item Felelősség\\
A grafikus motorhoz szükséges interfész. Olyan osztályok, melyek kirajzolható elemeket tartalmaznak megvalósítják ezt az interfészt.
\item Metódusok
	\begin{itemize}
		\item void \textbf{paint}(Graphics2D g): Rajzolást elvégző metódus.
	\end{itemize}
\end{itemize}

\subsection{List}
\begin{itemize}
    \item Felelősség
        \begin{itemize}
        \item Objektumok tárolása, ezt az interfészt megvalósító osztályban.
        \item \url{https://docs.oracle.com/javase/6/docs/api/java/util/List.html}
        \end{itemize}
\end{itemize}

\subsection{MapBuilder}
\begin{itemize}
\item Felelősség\\
A pálya felépítéséért, a checkpointok tárolásáért és a robot pályán tartózkodásának vizsgálatáért felelős osztály.
\item Attribútumok
	\begin{itemize}
		\item \textbf{List} checkpoints: Tárolja a checkpointokat reprezentáló objektumokat List adatszerkezetben.
		\item \textbf{Object} map: A pályát reprezentáló objektum.
		\item \textbf{int[]} startPosPlayerOne: Meghatároz egy (x,y) koordinátát, ahol az első játékos kezd.
		\item \textbf{int[]} startPosPlayerTwo: Meghatároz egy (x,y) koordinátát, ahol az második játékos kezd.
	\end{itemize}
\item Metódusok
	\begin{itemize}
		\item \textbf{MapBuilder}(): Konstruktor, a pálya beolvasása fájlból, majd létrehozása.
		\item int[] \textbf{getStartPosPlayer}(int id): Paraméterül kap egy Robot id-t, majd visszatér egy int tömbbel, melyben található a robot kezdőpozíciója a pályán.
		 \item boolean \textbf{obstacleOutsideOfMap}(Obstacle obs): Egy akadályt vizsgál, hogy a pályán van-e.
		\item boolean \textbf{robotOutsideOfMap}(Robot r): Igaz értéket ad vissza, ha a robot leesett a pályáról, hamisat ha még rajta van.
	\end{itemize}
\end{itemize}

\subsection{Obstacle}
\begin{itemize}
\item Felelősség\\
A pályán/játékosoknál lévő különböző akadályokat (ragacs,olaj) összefogó ősosztály.
\item Ősosztályok\\
Unit
\item Attribútumok
	\begin{itemize}
		\item \textbf{int} lifetime: Megmondja, hogy hány kör óta lett letéve az akadály.
	\end{itemize}
		\item \textbf{int} HEIGHT: Az akadályokat jellemző hosszúság. Szükség van rá, hogy létrehozzuk a leszármazottak hitbox-át(sokszög pályaelem).
		\item \textbf{int} WIDTH: Az akadályokat jellemző szélesség. Szükség van rá, hogy létrehozzuk a leszármazottak hitbox-át(sokszög pályaelem).
		
\item Metódusok
	\begin{itemize}
		\item \textbf{Obstacle}(int x, int y, String imagelocation): Meghívja a Unit konstruktorát a megadott adatokkal és létrehoz egy sokszög elemet ami reprezentálja a pályán majd.
		\item void \textbf{effect}(Robot r): Meghatározza, milyen hatással van a robotra, ha érintkezik egy Obstacle-lel. Absztrakt metódus.
		\item void \textbf{move}(): Mozgatásért felelős függvény. Ősosztályból öröklött, felülírt függvény. Mivel nem lehetséges az akadályok mozgása, ezért üres a függvény törzse.
	\end{itemize}
\end{itemize}

\subsection{Oil}
\begin{itemize}
\item Felelősség\\
A pályára lerakható olaj megvalósítása. Ha belelép egy játékos egy ilyen olajfoltba, az effect függvény letiltja a mozgatást az adott roboton a következő ugrásig.
\item Ősosztályok\\
Unit $\rightarrow$ Obstacle 
\item Metódusok
	\begin{itemize}
		\item \textbf{Oil}(int x, int y, String imagelocation): Egy Oil elem létrehozásáért felelős.
		\item void \textbf{effect}(Robot r): Meghatározza, milyen hatással van a robotra, ha beleugrik egy olajfoltba. Ebben az esetben letiltja a játékost, hogy irányt váltson.
	\end{itemize}
\end{itemize}

\subsection{Phoebe}
\begin{itemize}
\item Felelősség\\
A játék motorját képviselő osztály. A robotok pozíciójáért, az akadályok elhelyezéséért és a játék végéért felel.
\item Attribútumok
	\begin{itemize}
		\item \textbf{boolean} ended: Állapot változó, ha vége a játéknak, akkor true. Ha beteljesül egy játék végét jelentő esemény, akkor ezen a változón keresztül leáll a játék és megállapítódik a nyertes.
		\item \textbf{Settings} gameInfo: A játék beállításait tartalmazó osztály referenciája.
		\item \textbf{MyTimer} gameTimer: A játékhoz tartozó számlálóra mutató referencia.
		\item \textbf{HUD} hud: A játékosok előrehaladását, ragacs és olajkészleteit tartja számon.
		\item \textbf{MapBuilder} map: A pályát reprezentáló objektum. Tárolja még a checkpointokat és a robotok kezdő koordinátáit.
		\item \textbf{List<Obstacle>} obstacles: A játékban szereplő akadályok listája.
		\item \textbf{List<Robot>} robots: A játékban szereplő robotok listája.
	\end{itemize}
\item Metódusok
	\begin{itemize}
		\item \textbf{Phoebe}(Setting set): A játék felépítése, a robotok lista, az akadályok lista létrehozása.
		\item void \textbf{addObstacle}(Obstacle ob): Az obstacles tárolóba helyez egy akadályt.
		\item void \textbf{init}(): Inicializáló függvény. Feladat, hogy robotokat, akadályokat készítsen a játék kezdete előtt, létrehozza a pályát.
		\item void \textbf{paint}(Graphics2D g): Rajzoló függvény.
		\item void \textbf{run}(): Ez a metódus futtatja a főciklust, amelyben maga a játék működik.
	\end{itemize}
\end{itemize}

\subsection{Robot}
\begin{itemize}
\item Felelősség\\
A játékban résztvevő robotok viselkedését és kezelését leíró osztály.
\item Interfészek\\
IVisible, Unitból származva
\item Ősosztályok\\
Unit
\item Attribútumok
	\begin{itemize}
	    \item \textbf{int} arrowendx: A robot irányítását segítő nyílnak az x koordinátája, a nyíl kirajzolásánál van szerepe.
		\item \textbf{int} arrowendy: A robot irányítását segítő nyílnak az y koordinátája, a nyíl kirajzolásánál van szerepe.
		\item \textbf{double} alpha: A robot irányítását segítő nyíl vízszintessel bezárt szöge. A nyil kirajzolásánál, az ugrás végpontjának meghatározánál van szerepe.
		\item \textbf{int} HEIGHT: A robot képének magassága, collision                      detektálásnál, továbbá az irányítást segítő nyíl kezdő koordinátájának                  meghatározásánál szükséges.
		\item \textbf{int} WIDTH: A robot képének szélessége, funkcionalitásban hasonló a WIDTH-hez.
    	\item \textbf{int} ID: A robot példányának egyedi azonosítója, a keyconfig sorának indexelésére szükséges.
    	\item \textbf{boolean} leftobstacle: Logikai változó, tárolja, hogy az adott körben raktunk-e le már akadályt.
    	\item \textbf{boolean} moved: Azt jelöli, hogy lépett-e már a robot az aktuális körben. A megjelenítésnél (a nyilat ugrás közben nem jelenítjük meg), illetve az irányítás letiltásánál van szerepe (olajba lépés esetén).
		\item \textbf{int} numGlue: A robot számára felhasználható ragacsok száma.
	    \item \textbf{int} numOil: A robot számára felhasználható olajok száma.
	    \item \textbf{boolean} oiled: Azt jelzi, hogy olajba lépett-e. Ennek hatására a mozgás iránya módosíthatatlanná válik egy kis időre.
    	\item \textbf{Phoebe} p: Referencia a játékmotorra.
    	\item \textbf{int} r: A robot következő ugrását mutató nyíl hossza
    	\item \textbf{double} slowed: A sebesség módosításáért felel, default értéke 1.0. Amennyiben ragacsba lép a robot ez 0.5-re módosul és minden ugrás végén visszaáll az eredeti értékére. Ugrásnál ezzel szorozzuk be a végkordinátát kiszámító sugár hosszát.
		\item \textbf{int} staticID: Az osztályhoz tartozó statikus azonosító, a példány azonosítójának(id) meghatározásához szükséges.
\end{itemize}
\item Metódusok

	\begin{itemize}
		\item \textbf{Robot}(int x,int y,String imagelocation,Phoebe p): Létrehoz egy robotot a megadott x,y koordinátákon, betölti a képét az imagelocation cím alapján, eltárolja a játékmotor referenciáját, továbbá inicializálja felhasználható akadályok számát.
		\item void \textbf{bounce}(): Két robot ütközése után meghívódó függvény. Az ütközés függvényében kiszámolja a lepattanás irányát és letiltja egy körre az irány változtatást.
		\item boolean \textbf{collisionWithObstacle}(Obstacle o): Ellenőrzi hogy a robot ütközött-e az akadállyal. Igazzal tér vissza ha igen, hamissal ha nem.
		\item void \textbf{collisionWithRobot}(Robot r): Ellenőrzi, hogy a robot ütközött-e másik robottal. Ha igen, akkor gondoskodik róla, hogy a robotok a megfelelő szögben pattanjanak le egymásról.
		\item void \textbf{deathanimation}(): A Robot halálának grafikus megjelenítéséért felelős függvény.
		\item int \textbf{getNumGlue}(): Visszatér a felhasználható ragcsok számával.
		\item int \textbf{getNumOil}(): Visszatér a felhasználható olajok számával.
		\item void \textbf{incNumGlue}(): Növeli a robotnál tárolt ragacsok számát.
		\item void \textbf{incNumOil}(): Növeli a robotnál tárolt olajok számát.
		\item void \textbf{keyPressed}(int e): A robot irányítását megvalósító függvény, a játékmotor keylistener-e által hívódik meg, a lenyomott billentyű keyevent-jére. A következő ugrás beállítása, a ragacs/olaj lerakása történhet itt.
		\item void \textbf{move}(): A robot mozgását megvalósító függvény.
		\item void \textbf{paint}(Graphics2D g): Rajzoló függvény.
		\item void \textbf{setOiled}(): Átállítja a robot olaj effekt követéséhez tartozó állapot változót.
		\item void \textbf{setGlue}(): Átállítja a robot ragacs effekt követéséhez tartozó állapot változót.

	\end{itemize}
\end{itemize}

\subsection{MyTimer}
\begin{itemize}
\item Felelősség\\
A játék elején a kezdésig visszaszámol. Játéktípustól függően felfelé vagy visszafelé számol. Ez az osztály felelős, hogy ha lejárt az idő, akkor legyen vége a játéknak.
\item Metódusok
	\begin{itemize}
		\item \textbf{MyTimer}(int i): Konstruktor; inicializál egy viszaszámláló órát.
		\item boolean \textbf{isZero}(): Igazzal tér vissza, ha a megadott idő lejárt.
		\item int \textbf{getTime}(): Visszaadja az eltelt időt másodpercben.
		\item void \textbf{start}(): Elindítja / újraindítja a számlálót.
	\end{itemize}
\end{itemize}

\subsection{Unit}
\begin{itemize}
\item Felelősség\\
A pályán található objektumokért felel és azok viszonyáról (például ütközésükről).
\item Attribútumok
	\begin{itemize}
		\item \textbf{Object} hitbox: Az egységet a pályán reprezentáló sokszög.
		\item \textbf{int} x: Az egység x koordinátája
		\item \textbf{int} y: Az egység y koordinátája
	\end{itemize}
\item Metódusok
	\begin{itemize}
	    \item \textbf{Unit}(): A Unit osztály konstruktora. Feladata, hogy eltárolja az x,y koordinátát.
		\item boolean \textbf{intersect}(Unit u): Két egység ütközését meghatározó függvény.
		\item void \textbf{move}() : Absztrakt függvény, mely a leszármazottakban fog megvalósulni. Az egységek mozgásáért felelős.
	\end{itemize}
\end{itemize}

\setcounter{chapter}{5}
\chapter{Skeleton beadás}
\section{Fordítási és futtatási útmutató}

\subsection{Fájllista}

\begin{fajllista}

\fajl
{LogBase.java} % Kezdet
{2069 byte} % Idptartam
{2015.03.23~00:23~} % Résztvevők
{Szkeleton felületének kialakítása} % Leírás

\fajl
{MapBuilder.java}
{2455 byte}
{2015.03.23~00:23~}
{MapBuilder osztály}

\fajl
{Glue.java}
{1476 byte}
{2015.03.23~00:23~}
{Glue osztály}

\fajl
{HUD.java}
{5193 byte}
{2015.03.23~00:23~}
{HUD osztály}

\fajl
{Obstacle.java}
{2316 byte}
{2015.03.23~00:23~}
{Obstacle osztály}

\fajl
{Oil.java}
{1593 byte}
{2015.03.23~00:23~}
{Oil osztály}

\fajl
{Phoebe.java}
{9491 byte}
{2015.03.23~00:23~}
{Phoebe osztály, Settings osztály}

\fajl
{Robot.java}
{10649 byte}
{2015.03.23~00:23~}
{Robot osztály}

\fajl
{Unit.java}
{1846 byte}
{2015.03.23~00:23~}
{Unit osztály}

\fajl
{\vtop{\hbox{\strut scrumthat\_skeleton\_}\hbox{\strut eclipseproject.jar}}
}
{15366 byte}
{2015.03.23~10:20~}
{Importálható eclipse archívum}

\end{fajllista}

\subsection{Fordítás}

1. Nyissa meg az Eclipse IDE-t\\
2. Állítsa be a következőt: \\
Window -> Preferences -> Workspace -> Text file encoding -> Other(UTF-8)\\
3. File -> Import -> General -> Existing Project into Workspace-> \\Select archive file: scrumthat\_skeleton\_eclipseproject.jar\\
4. Finish

\subsection{Futtatás}
Eclipse környezetben:\\
Run -> Run As -> Java Application

\section{Értékelés}

\begin{ertekeles}
\tag{Kovács} % Tag neve
{22}        % Munka szazalekban
\tag{Lovász}
{20}
\tag{Graics}
{17}
\tag{Magyar}
{20}
\tag{Tóth}
{21}
\end{ertekeles}


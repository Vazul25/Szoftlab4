% Szglab4
% ===========================================================================
%
\chapter{Prototípus beadása}

\thispagestyle{fancy}

\section{Fordítási és futtatási útmutató}

\subsection{Fájllista}

\begin{fajllista}

\fajl
{Scrum\_that\_proto\_test.zip} % Kezdet
{775 213 byte} 
{2015.04.19. ~21:46~} 
{Tesztrendszer} 

\fajl
{Scrum\_that\_proto\_teljes.zip} % Kezdet
{676 528 byte} 
{2015.04.20. ~00:01~} 
{Prototípus} 

\end{fajllista}

\subsection{Tesztelés}

1. Csomagolja ki a Scrum\_that\_proto.zip állományt.\\
2. Lépjen be a Prototipus gyökérkönyvtárba.\\
3. Indítsa el a Clean.bat futtatható scriptet.\\
4. Indítsa el a Tesztelo.bat futtatható scriptet.\\
5. Figyelje a megjelenő parancssort és ha szükséges nyomjon meg egy gombot.\newline

A tesztek elvárt és eredményezett kimeneteit ellenörző script a következőképp épül fel:\newline
-----TesztX------\\
<compare>\\
\comment{0 vagy több sor kimeneti nyelv}\\
</compare>\\

Ha a compare kulcsszavak között 0 sor van akkor a Teszt sikeresnek mondható, ha nem akkor az eltérést látható ott. Ekkor a teszt sikertelen.

\subsection{Prototípus futtatás}
Eclipse környezetben:\\
1. Importálja a projektet a Scrum\_that\_proto\_teljes.zip archív fájlból.\\
2. Run As: Java Application

\section{Tesztek jegyzőkönyvei}

\subsection{CollisionWithRobot\_VOLTÜTKÖZÉS\_TESZT}

\tesztok{Tóth Krisztián Dávid}{2015.04.19. 22:00}

\subsection{CollisionWithRobot\_NEMVOLTÜTKÖZÉS\_TESZT}

\tesztok{Tóth Krisztián Dávid}{2015.04.19. 22:00}

\subsection{CollisionWithRobot\_IRÁNYVÁLTOZTATÁS\_TESZT}

\tesztok{Tóth Krisztián Dávid}{2015.04.19. 22:00}

\subsection{CollisionWithObstacles\_OLAJBA.UGRÁS\_TESZT }

\tesztok{Tóth Krisztián Dávid}{2015.04.19. 22:00}

\subsection{CollisionWithObstacles\_RAGACSBA.UGRÁS\_TESZT}

\tesztok{Tóth Krisztián Dávid}{2015.04.19. 22:00}

\subsection{ CollisionWithObstacles\_OLAJ.HATÁSA\_TESZT}

\tesztok{Tóth Krisztián Dávid}{2015.04.19. 22:00}

\subsection{CollisionWithObstacles\_RAGACS.HATÁSA\_TESZT }

\tesztok{Tóth Krisztián Dávid}{2015.04.19. 22:00}

\subsection{robotOutsideOfMap\_A.PÁLYÁRÓL.LEESETT\_TESZT}

\tesztok{Tóth Krisztián Dávid}{2015.04.19. 22:00}

\subsection{robotOutsideOfMap\_NEM.ESETT.LE.PÁLYÁRÓL\_TESZT}

\tesztok{Tóth Krisztián Dávid}{2015.04.19. 22:00}

\subsection{checkpointSearch\_CHEICKPOINTONBA.UGRÁS\_TESZT}

\tesztok{Tóth Krisztián Dávid}{2015.04.19. 22:00}

\subsection{checkpointSearch\_CHEICKPOINTONBA.NEM.UGRÁS\_TESZT}

\tesztok{Tóth Krisztián Dávid}{2015.04.19. 22:00}

\subsection{RobotCollisionWithCleaner\_TESZT}

\tesztok{Tóth Krisztián Dávid}{2015.04.19. 22:00}

\subsection{Initialisation\_Test}

\tesztok{Tóth Krisztián Dávid}{2015.04.19. 22:00}

\subsection{GameEndWithTimeElapsing\_Test}

\tesztok{Tóth Krisztián Dávid}{2015.04.19. 22:00}

\subsection{AddObstacle\_AKADÁLY\_LERAKÁS\_TESZT}

\tesztok{Tóth Krisztián Dávid}{2015.04.19. 22:00}

\subsection{ObstacleLife\_AKADÁLY\_ÉLETTARTAM\_TESZT}

\tesztok{Tóth Krisztián Dávid}{2015.04.19. 22:00}

\subsection{Cleaner\_TAKARÍTÓ\_KISROBOT\_MOZGÁS\_TAKARÍTÁS\_TESZT}

\tesztok{Tóth Krisztián Dávid}{2015.04.19. 22:00}


\pagebreak
\section{Értékelés}

\begin{ertekeles}
\tag{Kovács} % Tag neve
{20}        % Munka szazalekban
\tag{Lovász}
{20}
\tag{Graics}
{20}
\tag{Magyar}
{20}
\tag{Tóth}
{20}
\end{ertekeles}


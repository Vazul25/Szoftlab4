% Szglab4
% ===========================================================================
%
\setcounter{chapter}{-1}

\chapter{Módosítások}

\subsection{Bemeneti nyelv}

\begin{itemize}
\item Robot
    \begin{itemize}
	\item Leírás: Robot létrehozása a pályán megadott koordinátákon.
	\item Opciók: Két koordináta.
	\end{itemize}
	
\item Cleaner
    \begin{itemize}
	\item Leírás: Cleaner létrehozása a pályán megadott koordinátákon.
	\item Opciók: Két koordináta.
	\end{itemize}
\end{itemize}

\chapter{Részletes tervek}

\thispagestyle{fancy}

\section{Osztályok és metódusok tervei}

\subsection{Osztály1}
\begin{itemize}
\item Felelősség\newline
\comment{Mi az osztály felelőssége. Kb 1 bekezdés. Ha szükséges, akkor state-chart is.}
\item Ősosztályok\newline
\comment{Mely osztályokból származik (öröklési hierarchia)\newline
Legősebb osztály $\rightarrow$ Ősosztály2 $\rightarrow$ Ősosztály3...}
\item Interfészek\newline
\comment{Mely interfészeket valósítja meg.}
\item Attribútumok\newline
\comment{Milyen attribútumai vannak}
	\begin{itemize}
		\item attribútum1: attribútum jellemzése: mire való, láthatósága (UML jelöléssel), típusa
		\item attribútum2: attribútum jellemzése: mire való, láthatósága (UML jelöléssel), típusa
	\end{itemize}
\item Metódusok\newline
\comment{Milyen publikus, protected és privát  metódusokkal rendelkezik. Metódusonként precíz leírás, ha szükséges, activity diagram is  a metódusban megvalósítandó algoritmusról.}
	\begin{itemize}
		\item int foo(Osztály3 o1, Osztály4 o2): metódus leírása, láthatósága (UML jelöléssel)
		\item int bar(Osztály5 o1): metódus leírása, láthatósága (UML jelöléssel)
	\end{itemize}
\end{itemize}

\subsection{Osztály2}
\begin{itemize}
\item Felelősség\newline
\comment{Mi az osztály felelőssége. Kb 1 bekezdés. Ha szükséges, akkor state-chart is.}
\item Ősosztályok\newline
\comment{Mely osztályokból származik (öröklési hierarchia)\newline
Legősebb osztály $\rightarrow$ Ősosztály2 $\rightarrow$ Ősosztály3...}
\item Interfészek\newline
\comment{Mely interfészeket valósítja meg.}
\item Attribútumok\newline
\comment{Milyen attribútumai vannak}
	\begin{itemize}
		\item attribútum1: attribútum jellemzése: mire való, láthatósága (UML jelöléssel), típusa
		\item attribútum2: attribútum jellemzése: mire való, láthatósága (UML jelöléssel), típusa
	\end{itemize}
\item Metódusok\newline
\comment{Milyen publikus, protected és privát  metódusokkal rendelkezik. Metódusonként precíz leírás, ha szükséges, activity diagram is  a metódusban megvalósítandó algoritmusról.}
	\begin{itemize}
		\item int foo(Osztály3 o1, Osztály4 o2): metódus leírása, láthatósága (UML jelöléssel)
		\item int bar(Osztály5 o1): metódus leírása, láthatósága (UML jelöléssel)
	\end{itemize}
\end{itemize}

\section{Objektum katalógus}

\subsection{Glue}
A „Glue” objektum megvalósít egy adott tulajdonságú akadályt. Amely robot belemegy, annak a sebességét megfelezi. 
\subsection{GUI}
A grafikus felületet megvalósító objektum. Ez az objektum maga a menü, ami a játék indítása után ugrik fel. Itt találhatóak a beállítások (mint például a gondolkodás idő és a maximális játék idő vagy a körök száma) és a játékmódok. Gombnyomásra fogja elindítani a játék működési szálát. Ez az objektum kezeli az ablak eseményeit és a játék bezárását.
\subsection{HUD}
Ez az objektum követi és nyilvántartja, hogy a robotok hány checkpoint-on mentek át, illetve kiírja a képernyőre a hátramaradó időt és a megtett körök számát. Feladata, hogy minden körben megvizsgálja, hogy a robotok elérték-e a következő checkpointot.
\subsection{MapBuilder}
Fájlból beolvassa és létrehozza a memóriában a pályát, a kezdő pozíciókat és a checkpointokat reprezentáló objektumokat.  Mivel a  MapBuilder objektum tárolja a pályát így feladat, hogy vizsgálja a robotok, akadályok azon belül tartózkodását.  
\subsection{Oil}
Ez az objektum az Obstacle osztály leszármazottja. Hasonlóan a Glue objektumhoz, egy adott hatást valósít meg, ami letiltja a következő körben történő irányítását a robotnak, ami belelépett.
\subsection{Phoebe}
A játék logikát megvalósító objektum. Listában tárolja a pályán tartózkodó robotokat, akadályokat és figyeli, hogy mikor ér véget a játék. A „Phoebe” objektum rajzolja ki az objektumokat a pályán és szálként indítható osztályt, melyben maga a játék fut. Játékindításkor berakja a pályára a robotokat és az akadályokat a kezdő pozíciókba. Ebben az objektumban történnek az ellenőrzések (akadályba vagy robotba ütközések, pályáról leesés).
\subsection{Robot}
Olyan objektum, mely a pályán található robotokat valósítja meg. Leírja a viselkedésüket és a kezelésüket. A „Robot” osztály a Unit-ból származik le, ezáltal van pozíciója és az ütközés is le van kezelve. Felelős a mozgásért, megállapítja egy adott akadállyal vagy robottal ütközött-e és kezeli a robot által felhasználható akadálykészleteket, illetve tartalmaz gombnyomást lekezelő metódusokat is.
\subsection{MyTimer}
Az eltelt időt és a fennmaradt idő nyilvántartásáért felelős. Ilyen például a játék elején a három másodperces visszaszámlálás vagy az időlimites játékmód esetén, amikor a maximális időtől számol visszafelé.

\subsection{MyListener}
A játék keylistener-jét megvalosító osztály. Külön szálon fut, hogy az egyszerre lenyomott gombok ne okozhasanak problémát. A játékban részvevő robotok KeyPressed függvényét hivogatja, a megfelelő KeyEvent paraméterrel.
\subsection{Cleaner}
\comment{...}

\section{Statikus struktúra diagramok}

\begin{figure}[h]
\begin{center}
\includegraphics[width=17cm]{images/struktdiagram.PNG}
\caption{Statikus struktúra diagram}
\label{fig:example3}
\end{center}
\end{figure}
\pagebreak


\section{Osztályok leírása}

\subsection{Cleaner}
\begin{itemize}
\item Felelősség\\

\item Ősosztályok\\
Unit$\rightarrow$Robot
\item Attribútumok
    \begin{itemize}
        \item List obstacles:  
    \end{itemize}
\item Metódusok
	\begin{itemize}
	    \item \textbf{Cleaner}(int x, int y, Phoebe p):
		\item void \textbf{setObstacles}(List obsts): 
		\item boolean \textbf{collisionWithRobot}():
		\item void \textbf{collisionWithCleaner}():
		\item void \textbf{move}():
	\end{itemize}
\end{itemize}



\subsection{GUI}
\begin{itemize}
\item Felelősség\\
A grafikus felületért felelős osztály, amely a menüt és a játékot jeleníti meg.
\item Attribútumok
	\begin{itemize}
		\item \textbf{Phoebe} game: referencia a játékra
	\end{itemize}
\item Metódusok
	\begin{itemize}
		\item\textbf{GUI}(): Konstruktor. Beállítja az ablak nevét, létrehozza az ablak elemeit, elrendezi őket és beállítja a figyelőket(ActionListener).
	\end{itemize}
\end{itemize}

\subsection{HUD}
\begin{itemize}
\item Felelősség\\
A robotok megtett köreit és checkpontjait tartja számon. Megvalósítja a checkpoint ellenőrzést.
\item Attribútumok
	\begin{itemize}
		\item \textbf{int[]} checkpointReached: Minden robothoz külön tárolja a legutoljára érintett checkpoint sorszámát.
		\item \textbf{int[]} lap: Minden robothoz tárolja a megtett körök számát. 
		\item \textbf{List} checkpoints: Tárolja a checkpointokat reprezentáló objektumokat List adatszerkezetben. A checkpointSearch függvény kérdezi le ebből a következő checkpoint helyzetét. 
		\item \textbf{List<Robot>} robots: A robotokat tároló List adatszerkezet. A checkpointSearch függvény kérdezi le ebből a robotokat, majd azok helyzetét.
	\end{itemize}
\item Metódusok
	\begin{itemize}
		\item \textbf{HUD}(List<Robot> robs): Robot objektumokat tároló ArrayList. Célja, hogy a checkpointsearch() függvényben minden robotra elvégezzük a keresést.
		\item void \textbf{checkpointSearch}(): Minden híváskor ellenőrzi, hogy a robot és a checkpoint metszete üres-e. 
		\item int \textbf{endOfTheGame}(): A játék végén eldönti, hogy melyik játékos nyert. Visszatér egy számmal, amiből egyértelműen eldönthető, hogy ki nyert. Ha negatív akkor az 1-es számú játékos nyert, ha nulla akkor döntetlen, ha pozitív akkor a 2-es számú játékos nyert.
		\item void \textbf{setCheckpointReached}(Robot r): Ha a paraméterként átadott robot következő checkpointja a célvonal (utolsó checkpoint) akkor lenullázza a checkpointReached-et és növeli a megtett körök számát, illetve ha nem akkor növeli az érintett checkpointok számát.
		\item void \textbf{setCheckpoints}(List checkObj): Checkpointokat reprezentáló adatszerkezet betöltése.CheckpointReached inicializálása a checkpointok számától függően.
	\end{itemize}
\end{itemize}

\subsection{IVisible}
\begin{itemize}
\item Felelősség\\
A grafikus motorhoz szükséges interfész. Olyan osztályok, melyek kirajzolható elemeket tartalmaznak megvalósítják ezt az interfészt.
\item Metódusok
	\begin{itemize}
		\item void \textbf{paint}(Graphics2D g): Rajzolást elvégző metódus.
	\end{itemize}
\end{itemize}

\subsection{List}
\begin{itemize}
    \item Felelősség
        \begin{itemize}
        \item Objektumok tárolása, ezt az interfészt megvalósító osztályban.
        \item \url{https://docs.oracle.com/javase/6/docs/api/java/util/List.html}
        \end{itemize}
\end{itemize}

\subsection{MapBuilder}
\begin{itemize}
\item Felelősség\\
A pálya felépítéséért, a checkpointok tárolásáért és a robot pályán tartózkodásának vizsgálatáért felelős osztály.
\item Attribútumok
	\begin{itemize}
		\item \textbf{List} checkpoints: Tárolja a checkpointokat reprezentáló objektumokat List adatszerkezetben.
		\item \textbf{Object} map: A pályát reprezentáló objektum.
		\item \textbf{int[]} startPosPlayerOne: Meghatároz egy (x,y) koordinátát, ahol az első játékos kezd.
		\item \textbf{int[]} startPosPlayerTwo: Meghatároz egy (x,y) koordinátát, ahol az második játékos kezd.
	\end{itemize}
\item Metódusok
	\begin{itemize}
		\item \textbf{MapBuilder}(): Konstruktor, a pálya beolvasása fájlból, majd létrehozása.
		\item int[] \textbf{getStartPosPlayer}(int id): Paraméterül kap egy Robot id-t, majd visszatér egy int tömbbel, melyben található a robot kezdőpozíciója a pályán.
		 \item boolean \textbf{obstacleOutsideOfMap}(Obstacle obs): Egy akadályt vizsgál, hogy a pályán van-e.
		\item boolean \textbf{robotOutsideOfMap}(Robot r): Igaz értéket ad vissza, ha a robot leesett a pályáról, hamisat ha még rajta van.
	\end{itemize}
\end{itemize}

\subsection{MyListener}
\begin{itemize}
\item Felelősség\\
A külön szálon futó KeyListener-t megvalósító osztály.
\item Interface-ek:
    \begin{itemize}
    \item KeyListener
    \item Runnable
    \end{itemize}
\item Attribútumok
	\begin{itemize}
		\item -\textbf{boolean} isUp: Azt tárolja, hogy le van-e nyomva a felfele nyil.
    	\item -\textbf{boolean} isDown: Azt tárolja, hogy le van-e nyomva a lefele nyil.
    	\item -\textbf{boolean} isRight:Azt tárolja, hogy le van-e nyomva a jobbra nyil.
    	\item -\textbf{boolean} isLeft:Azt tárolja, hogy le van-e nyomva a balra nyil.
    	\item -\textbf{boolean} isW:Azt tárolja, hogy le van-e nyomva a W a billentyűzeten.
    	\item -\textbf{boolean} isD:Azt tárolja, hogy le van-e nyomva a D a billentyűzeten.
    	\item -\textbf{boolean} isS:Azt tárolja, hogy le van-e nyomva az S a billentyűzeten.
    	\item -\textbf{boolean} isA:Azt tárolja, hogy le van-e nyomva az A a billentyűzeten.
   	\item -\textbf{List<Robot>} robots: Azon robotok listája akiknek a keyPressed függvényét kell hívnia.
	
	
	\end{itemize}
\item Metódusok
	\begin{itemize}
		\item+ \textbf{MyListener}(List<Robot> r): beállítja a lenyomott gombokat figyelő változókat false-ra , továbbá beállítja a robots változó referenciáját a paraméterként kapotra.
			\item+ void\textbf{keyPressed}(KeyEvent e): Beállítja a lenyomot gombokat figyelő változók küzül a Keyeventnek megfelelőt  true-ra , ha meg nyomták valamelyiket  a figyelt gombok közül.
			\item+ void\textbf{keyReleased}(KeyEvent e): Beállítja a lenyomot gombokat figyelő változók küzül a Keyeventnek megfelelőt false-ra , ha fel engedték valamelyiket a figyelt, lenyomott gombok közül.
				\item+ void\textbf{run}(): Végtelen ciklust futtat. Megvizsgálja hogy melyik gombok vannak lenyomva, majd a nekik megfelelő KeyEvent-tel paraméterezve meghívja a hozzátartozó robotoknak a Keypressed függvényét. Majd alszik a szál 30 mili secundomig.
	\end{itemize}
\end{itemize}

\subsection{MyTimer}
\begin{itemize}
\item Felelősség\\
A játék elején a kezdésig visszaszámol három másodpercet, utána indulhat a játék. Játéktípustól függően felfelé(kör játékmód) vagy visszafelé(idő játékmód) számol. Ez az osztály felelős, azért ha lejár az idő vége legyen a játéknak,
\item Attribútumok
    \begin{itemize}
        \item - enum DIR: Az óra számolási irányának enumerizácciója.
        \item - long T\_start: Az óra indításának időpontja millisec pontosággal.
	    \item - int duration: Ha az óra visszafelé számol, akkor tárolja, hogy mennyi volt a kezdő érték, ha felfelé számol, akkor értéke 0. Mértékegysége millisekundum.
	    \item - DIR direction: Az óra számolási irányának eltárolásáért felelős enum.
    \end{itemize}
\item Metódusok
	\begin{itemize}
		\item + \textbf{MyTimer}(int i): Konstruktor. Ha 0-val vagy negatívval inicializálják felfele számol, ha pozitív számmal inicializálják akkor lefelé számol.
		\item + boolean \textbf{isZero}(): Az idő lejárását ellenörző függvény, megadja, hogy az indítás plusz a megadott időtartam kisebb-e a pillanatnyi időnél.
		\item + int \textbf{getTime}():  Ha pozitív száámal inicializálódott az objektum, akkor megadja mennyi idő van még hátra a visszaszámlálásból vagy, ha nullával, akkor a start() hívás óta eltelt idővel tér vissza.
		\item + void \textbf{start}(): Az óra indításakor vagy újraindításakor meghívott függvény. Csak akkor indul újra (visszaszámláló üzemmódban), ha elérte a 0-t. A Phoebe run() metódusa hívja meg, mikor vissza kell számolni a játék kezdete előtt három másodpercet. Illetve, a játék kezdetekor.
	\end{itemize}
\end{itemize}

\subsection{Obstacle}
\begin{itemize}
\item Felelősség\\
A pályán/játékosoknál lévő különböző akadályokat (ragacs,olaj) összefogó ősosztály.
\item Ősosztályok\\
Unit
\item Attribútumok
	\begin{itemize}
		\item \# \textbf{int} WIDTH: Az akadályokat jellemző szélesség. Szükség van rá, hogy létrehozzuk a leszármazottak hitbox-át(sokszög pályaelem).
		\item \# \textbf{int} HEIGHT: Az akadályokat jellemző hosszúság. Szükség van rá, hogy létrehozzuk a leszármazottak hitbox-át(sokszög pályaelem).
			\item \# \textbf{int} lifetime: Generikussan az akadályok életben maradásának ellenörzésére szolgál. Olaj esetében megadja, hogy hány kör telt el az akadály lerakása óta. Ragacs esetében pedig, hogy hányan léptek rá mióta lekerült. 
	\end{itemize}
\item Metódusok
	\begin{itemize}
		\item +\textbf{Obstacle}(int x, int y): meghívja a Unit konstruktorát a megadott adatokkal és létrehoz egy négyzet elemet ami reprezentálja a pályán majd.
		\item + void \textbf{effect}(Robot r): Meghatározza, milyen hatással van a robotra, ha érintkezik egy Obstacle-lel. Absztrakt.
		\item +boolean \textbf{checkAlive}():Az akadályok vizsgálata amit a játékmotor minden körben meghív minden akadályra. Absztrakt.
	\end{itemize}
\end{itemize}

\subsection{Glue}
\begin{itemize}
\item Felelősség\\
A játékban szereplő Ragacs foltok viselkedését leíró osztály
\item Ősosztályok\\
Unit$\rightarrow$Obstacle
\item Attribútumok
	\begin{itemize}
		\item -\textbf{BufferedImage} img: Ez a  statikus atribútum a ragacs képét tárolja,  a megjelenítésben van szerepe.
	\end{itemize}
\item Metódusok
	\begin{itemize}
		\item  \textbf{Glue}(int x,int y):A ragacs konstruktora, meghívja az őse (obstacles)                           konstruktorát x,y paraméterrel, továbbá beállítja a lifetime-ot 4-re.
		
		\item+ void \textbf{effect}(Robot r): Ütközéskor hívja meg az ütközést vizsgáló függvénye     a Robot osztálynak. Módosítja a robot slowed értékét a 50\%-ra a robot slowed attribútum         setterének meghívásával. Továbbá csökkenti a ragacs élettartalmát egyel.
	
		\item+ boolean \textbf{checkAlive}():A játékmotor hívja meg minden kör végén, ha a lifetime értéke>0 true-val tér vissza, különben false-al.
			\item+ void \textbf{paint}(Graphics2D g):Kirajzolja a ragacs képét az x,y koordinátákon.
				\item+ void \textbf{setUnitImage}():Beállítja a ragacs osztályhoz tartozó képet a user directoryban található glue.jpg-re
			\item+ String \textbf{toString}()Vissza ad egy stringet a  ragacs legfontosabb értékeivel.(x, y, WIDTH, HEIGHT, lifetime)
		
	\end{itemize}
\end{itemize}

\subsection{Oil}
\begin{itemize}
\item Felelősség\\
A pályára lerakható olaj megvalósítása. Ha belelép egy játékos egy ilyen olajfoltba az effect függvény letiltja a mozgatást az adott roboton a következő ugrásig.
\item Ősosztályok\\
Unit $\rightarrow$ Obstacle 
\item Attribútumok
	\begin{itemize}
		\item- \textbf{BufferedImage} img: Ez a  statikus atribútum az olaj képét tárolja,  a megjelenítésben van szerepe.
	\end{itemize}
\item Metódusok
	\begin{itemize}
		\item + \textbf{Oil}(int x, int y): Egy Oil elem létrehozásáért felelős. Meghívja az ős konstruktorát x,y-paraméterrel, továbbá beállítja a lifetime-ot default értékre.
		\item + void \textbf{effect}(Robot r): Meghatározza, milyen hatással van a robotra, ha beleugrik egy olajfoltba. Ebben az esetben letiltja a játékost, hogy irányt váltson.
			\item + void \textbf{setUnitImage}():Beállítja az olaj osztályhoz tartozó képet a user directoryban található oil.jpg-re
				\item + boolean \textbf{checkAlive}():A játékmotor hívja meg minden kör végén, lifetime értékét csökkenti 1 el, ha a lifetime értéke>0 (csökkentés után)true-val tér vissza, különben false-al.
		\item + String \textbf{toString}()Vissza ad egy stringet az olaj legfontosabb értékeivel.(x, y, WIDTH, HEIGHT, lifetime)
	\end{itemize}
\end{itemize}

\subsection{Phoebe}
\begin{itemize}
\item Felelősség\\
A játék motorját megvalósító objektum. Listában tárolja a pályán tartózkodó robotokat,kisrobotokat, akadályokat és 
 figyeli, hogy mikor ér véget a játék. A „Phoebe” objektum rajzolja ki az objektumokat a pályán és 
 szálként indítható osztályt, melyben maga a játék fut. Játékindításkor berakja a pályára a robotokat és 
 az akadályokat a kezdő pozíciókba. Ebben az objektumban történnek az ellenőrzések (akadályba vagy 
  robotba ütközések, pályáról leesés)
  \item Interface:
  \begin{itemize}
  \item Runnable
  \end{itemize}
\item Attribútumok
	\begin{itemize}
		\item- \textbf{boolean} ended: Állapot változó, ha vége a játéknak, akkor true. Ha beteljesül egy játék végét jelentő esemény, akkor ezen a változón keresztül leáll a játék és megállapítódik a nyertes.
		\item+ \textbf{BufferedImage} background: A játék hátterét adó kép.
		\item- \textbf{List<Robot>} robots: A játékban szereplő robotok listája.
		\item- \textbf{List<Obstacle>} obstacles: A játékban szereplő akadályok listája.
		\item- \textbf{HUD} hud: A játékosok előrehaladását, ragacs és olajkészleteit tartja számon
		\item- \textbf{MapBuilder} map: TODO
		\item- \textbf{Settings} gameInfo:A játék beállításait tartalmazza         \item- \textbf{List<Cleaner>} cleaners: A játékban lévő aktív kis tisztogató robotokat tartja számon.
        \item- \textbf{MyTimer} gameTimer: A játékban futó óra, ami visszaszámlálásoknál és a játék végének meghatározásánál játszik szerepet.

		
			
	\end{itemize}
\item Metódusok
	\begin{itemize}
		\item+ \textbf{Phoebe}(Setting set): A játék felépítése, a robotok,a tisztogató kisrobotok és  az akadályok listáinak létrehozása.Az ended inicializálása ,az init függvény meghívása és a grafikus felület felépítése történik itt.
		\item+ void \textbf{run}(): Ez a metódus futtatja a főciklust, amelyben maga a játék működik.
	\item+ void \textbf{Paint}(Graphics2D g2d): Kirajzolja a játék aktuális állását. 
	\item+ void \textbf{addObstacle}(Obstacle item): Hozzá ad egy Obstacle-t a játékban lévők listájához.
	
	\item- void \textbf{init}(): Inicializálja a játékot a kezdeti beállításokra. Létrehozza az időzítőt, a mapot, a robotokat a kezdőpozíciók szerint,a hudot és beállítja a különböző osztályokhoz tartozó statikus képeket, továbbá a pálya alap ragacsait és olajait is szétszórja.
	\end{itemize}
\end{itemize}

\subsection{Robot}
\begin{itemize}
\item Felelősség\\
A játékban résztvevő ugráló robotok viselkedését és kezelését leíró osztály, tárolja és kezeli a felhasználható akadályok számát.
  Olyan objektum, mely a pályán található robotokat valósítja meg. Leírja a viselkedésüket és a kezelésüket. 
  A „Robot” osztály a Unit-ból származik le, ezáltal van pozíciója és az ütközés is le van kezelve. 
  Felelős a mozgásért, megállapítja egy adott akadállyal vagy robottal ütközött-e és kezeli a felhasználó által leütött gombokat.
  \item Ősosztályok\\
Unit

\item Attribútumok
	\begin{itemize}
		\item\# \textbf{int} staticID: Az osztályhoz tartozó statikus azonosító, a példány                azonosítójának(id) meghatározásához szükséges.
			\item- \textbf{static final int} r: Az ugrás számításához tartozó sugár. 
				\item- \textbf{static final int} ANIMATIONSPEED: Az ugrás animálásának részletessége(hányszor hívja meg a paintet). 
		\item\# \textbf{static int} HEIGHT: A robot képének magassága, collision                      detektálásnál, továbbá az irányítást segítő nyíl kezdő koordinátájának                  meghatározásánál szükséges.
		\item\# \textbf{static int} WIDTH:A robot képének szélessége, funkcionalitásban hasonló a WIDTH-hez.
		\item- \textbf{int} ID: A robot példányának egyedi azonosítója, a keyconfig sorának                     indexelésére és a collison detektálásnál az önmagával való ütközés                      kivédésére szükséges.
		\item- \textbf{int} numGlue:A robotnál lévő ragacskészletet tárolja.
		\item- \textbf{int} numOil:A robotnál lévő olajkészletet tárolja.
		\item- \textbf{boolean} leftobstacle:Megmondja hogy raktunk-e már le ebben a körbe olajat vagy ragacsot kezdő érték false, minden lépés után vissza áll false ra és minden obstacle lerakásnál true ra .
	\item\# \textbf{BufferedImage}img[]:A robotok képeit tartalmazza,az animáció miatt többet.
	
		\item- \textbf{double} slowed: A sebesség modosításáért felel, default értéje 1.0, amennyiben ragacsba lép a robot ez 0.5-re módosul és minden ugrás végén visszaáll az eredeti értékére, ugrásnál ezzel szorozzuk be a végkordinátát kiszámító sugár hosszát.
		\item- \textbf{boolean} oiled: Azt jelzi, hogy olajba lépett-e, ennek hatására a mozgás iránya módosíthatatlanná válik egy kis időre. 
		\item\# \textbf{int} arrowendx: A robot irányítását segítő nyilnak az x koordinátája, a nyíl kirajzolásánál van szerepe.
		\item\# \textbf{int} arrowendy: A robot irányítását segítő nyilnak az y koordinátája, a nyíl kirajzolásánál van szerepe.
		\item- \textbf{double} alpha: A robot irányítását segítő nyíl vízszintessel bezárt szöge. A nyil kirajzolásánál, az ugrás végpontjának meghatározánál van szerepe.
		\item\# \textbf{boolean} moved: Azt jelöli, hogy lépett-e már a robot az aktuális körben. A megjelenítésnél(nyilat ugrás közben nem jelenítjük meg),illetve az irányítás letiltásánál van szerepe(olajba lépés esetén).
	
\end{itemize}
\item Metódusok\\
	\begin{itemize}
		\item+ \textbf{Robot}(int x,int y,Phoebe p): Létrehoz egy robotot a megadott x,y kordinátákon, inicializálja a tagváltozóit és eltárolja a játékmotor referenciáját.
		\item+ void\textbf{deathanimation}():A Robot halálának gafikus megjelenítéséért felelős függvény.
		\item+ void\textbf{setOiled}():Az oiled értékét true-ra állítja. 
		\item+ void\textbf{setGlued}():A slowed értékét 0.5-re állítja. 
		\item+ int\textbf{getId}():A Robot id-ét adja vissza.
		\item+ int\textbf{getNumGlue}():Visszatér a felhasználható ragcsok számával.
		\item+ int\textbf{getNumOil}():Visszatér a felhasználható olajok számával.
		\item+ void\textbf{incNumOil}():Növeli a robotnál tárolt olajok számát, ha az nem haladja meg a 3 at.
		\item+ void\textbf{incNumGlue}():Növeli a robotnál tárolt ragacsok számát, ha az nem haladja meg a 3 at.
	\item+ void\textbf{paint}(Graphics2D g):kirajzolja a robotot a saját koordinátáin, ha nem lép éppen akkor az irányítást segítő nyilat is.
	\item+ void\textbf{setUnitImage}():beállítja a robot osztályhoz tartozó képeket.
\item+ void\textbf{bounce}():A robotok ütközésekor a lepattanás céjának  koordinátáinak számolása történik itt.

		\item+ void\textbf{move}():A robot mozgatásáért és annak leanimálásáért felelős függvény.Kiszámolja az új koordinátát és oda ugrasztja a robotot, majd frissíti a hitboxot.
		\item+ boolean \textbf{collisionWithObstacle}(Obstacle o): Ellenőrzi hogy a robot ütközött-e az akadállyal, igazzal tér vissza ha igen, hamissal ha nem.
		\item+ boolean \textbf{collisionWithRobot}(Robot r): Ellenőrzi hogy a robot ütközött-e másik robottal ,  referencia alapján kiszüri ha önmagára hívják meg.Ha volt ütközés meghívja a bounce függvényt önmagára.Igazzal tér vissza ha volt ütközés és hamissal ha nem.
		\item+ void \textbf{keyPressed}(int e):A robot irányítását megvalósító függvény, a játékmotor keylistener-e által hívódik meg, a lenyomott billentyű azonosítójával. A következő ugrás beállítása, a ragacs/olaj lerakása történhet itt. A Settings.keyconfig változó felhasználásával.
				\item+ String \textbf{toString}()Vissza ad egy stringet a robot legfontosabb értékeivel.(id, slowed, oiled, x, y, nextx, nexty, alpha, WIDTH, HEIGHT, numGlue, numOil)
	\end{itemize}
\end{itemize}

\subsection{Unit}
\begin{itemize}
\item Felelősség\\
A pályán található objektumokért felel és azok viszonyáról (például ütközésükről).
\item Attribútumok
	\begin{itemize}
		\item \# \textbf{Object} hitbox: Az egységet a pályán reprezentáló sokszög.
		\item \# \textbf{int} x: Az egység x koordinátája
		\item \# \textbf{int} y: Az egység y koordinátája
	\end{itemize}
\item Metódusok
	\begin{itemize}
	    \item + \textbf{Unit}(): A Unit osztály konstruktora. Feladata, hogy eltárolja az x,y koordinátát.
		\item + boolean \textbf{intersect}(Unit u): Két egység ütközését meghatározó függvény.
		\item + void \textbf{move}() : Absztrakt függvény, mely a leszármazottakban fog megvalósulni. Az egységek mozgásáért felelős.
	\end{itemize}
\end{itemize}

\section{A tesztek részletes tervei, leírásuk a teszt nyelvén}
[A tesztek részletes tervei alatt meg kell adni azokat a bemeneti adatsorozatokat, amelyekkel a program működése ellenőrizhető. Minden bemenő adatsorozathoz definiálni kell, hogy az adatsorozat végrehajtásától a program mely részeinek, funkcióinak ellenőrzését várjuk és konkrétan milyen eredményekre számítunk, ezek az eredmények hogyan vethetők össze a bemenetekkel.]

\subsection{Teszteset1}
\begin{itemize}
\item Leírás\newline
\comment{szöveges leírás, kb. 1-5 mondat.}
\item Ellenőrzött funkcionalitás, várható hibahelyek
\item Bemenet\newline
\comment{a proto bemeneti nyelvén megadva (lásd előző anyag)}
\item Elvárt kimenet\newline
\comment{a proto kimeneti nyelvén megadva (lásd előző anyag)}
\end{itemize}

\subsection{CollisionWithRobot\_VOLTÜTKÖZÉS\_TESZT}
\begin{itemize}
	\item Leírás: Ez a teszt a collisionWithRobot(Robot r) függvény teszteléséért felelős.
			Létrehoz 2 robotot majd beállítja az őket, úgy hogy egymásra ugorjanak .
			Kiirja az adataikat  és meghivja a move függvényt ezt követően pedig az egyik collisionWithRobot függvényét a másikra	és újból ki listáza őket.	\newline
	\item Ellenőrzött funkcionalitás, várható hibahelyek: Azt ellenörizzük hogy sikeressen össze tudta-e vetni a hitboxokat a függvény az ugrás végeztével .
	Várható hiba ha hamissal tér vissza a függvény, amiből látjuk, hogy rossz a hitbox létrehozása az új helyen az ugrást követően.
	\item Bemeneti nyelv :
		\begin{itemize}
		\item Robot(500,500)
		\item Robot(600,400)
		\item Keypressed(Keyevent.VK\_A,90)
		\item listRobots
		\item move
		\item listRobots
		\end{itemize}

	\item Elvárt kimenet: \\
		"Robot [id=0,  slowed=1,oiled=false, x=500,y=500, 
		\\numGlue=3,numOil=3,nextx=500,
		\\nexty=400,alpha=1.57,width=40,height=40]"\newline
		"Robot [id=1,  slowed=1,oiled=false, x=600,y=400, 
		\\numGlue=3,numOil=3,nextx=600,
		\\nexty=300,alpha=1.57,width=40,height=40]"
		
		"nextx ,nexty modified to:500,400"
	
		"Robot [id=0,  slowed=1,oiled=false, x=500,y=500, 
		\\numGlue=3,numOil=3,nextx=500,
		\\nexty=400,alpha=1.57,width=40,height=40]"\newline
		"Robot [id=1,  slowed=1,oiled=false, x=600,y=400, 
		\\numGlue=3,numOil=3,nextx=500,
		\\nexty=400,alpha=1.57,width=40,height=40]"

		 "there was a collision between this: "Robot [id=0,  slowed=1,oiled=false, x=500,y=400, 
		\\numGlue=3,numOil=3,nextx=500,
		\\nexty=400,alpha=1.57,width=40,height=40]"\newline
		and this:
		"Robot [id=1,  slowed=1,oiled=false, x=500,y=400, 
		\\numGlue=3,numOil=3,nextx=500,
		\\nexty=150,alpha=3.04,width=40,height=40]"\newline
		
	    "Robot [id=0,  slowed=1,oiled=false, x=500,y=400, 
		\\numGlue=3,numOil=3,nextx=500,
		\\nexty=400,alpha=1.57,width=40,height=40]"\newline
		
		"Robot [id=1,  slowed=1,oiled=false, x=500,y=400, 
		\\numGlue=3,numOil=3,nextx=500,
		\\nexty=150,alpha=3.04,width=40,height=40]"\newline

\end{itemize}
\subsection{CollisionWithRobot\_NEMVOLTÜTKÖZÉS\_TESZT}
\begin{itemize}
	\item Leírás: Ez a teszt a collisionWithRobot(Robot r) függvény teszteléséért felelős.
			Létrehoz 2 robotot  .
			Kiirja az adataikat  és meghivja a move függvényt ezt követően pedig az egyik collisionWithRobot függvényét a másikra 
			és újból ki listáza őket.			\newline
	\item Ellenőrzött funkcionalitás, várható hibahelyek: Azt ellenörizzük hogy sikeressen össze tudta-e vetni a hitboxokat a függvény az ugrás végeztével .
	Várható hiba ha loggol collisiont a függvény, amiből látjuk, hogy rossz a hitboxok össze vetése , hisz nem lehet ütközés. 
	\comment{\\Ctrl-C, Ctrl-V hiba, most akkor mivel kell visszatérnie. \\ \\Kellene logolni a collisionWithRobot() kimenetelét, a függvényen belül? by Don.\\
	most úgy van a függvény hogy van visszatérési értéke ... így könnyebb lesz majd égetve logolni a teszten belül , de akár a belselyébe is be lehet rakni a kiirást( de multkor azt beszéltük hogy ha nincs log az is log (a collision kiirásának elmulasztása felér azzal hogy nem volt collision))}
	
	\comment{Hát nem tudom mi lenne jobb. Ennél a két tesztesetnél jó lenne látni, hogy megtörtént a vizsgálat és az eredményét.}
	
	\item Bemeneti nyelv :
		\begin{itemize}
		\item Robot(500,500)
		\item Robot(600,400)
		\item listRobots
		\item move
		\item listRobots
		\end{itemize}
	

	\item Elvárt kimenet: \\
		"Robot [id=0,  slowed=1,oiled=false, x=500,y=500, 
		\\numGlue=3,numOil=3,nextx=500,
		\\nexty=400,alpha=1.57,width=40,height=40]"\newline
		"Robot [id=1,  slowed=1,oiled=false, x=600,y=400, 
		\\numGlue=3,numOil=3,nextx=600,
		\\nexty=300,alpha=1.57,width=40,height=40]"
		
		
		\comment{most akkor merre ugrik? \\0.állapot Robot0(500,500)->(500,400),Robot1(600,400)->(600,300) \\1. állapot Robot0(500,400)->(400,400),Robot1(600,400)->(600,400)\\Figyelj arra is, hogy a nextx, nexty megfelelően értékbe álljanak, ne váltson irányt a Robot0. by Don}
		\comment {ehez volna jó ha működne a grafikus felület mert eléggé bele tudok zavarodni a monitor képpont indexelésébe}
		
		
		 "Robot [id=0,  slowed=1,oiled=false, x=500,y=400, 
		\\numGlue=3,numOil=3,nextx=500,
		\\nexty=400,alpha=1.57,width=40,height=40]"\newline
		"Robot [id=1,  slowed=1,oiled=false, x=600,y=300, 
		\\numGlue=3,numOil=3,nextx=600,
		\\nexty=300,alpha=1.57,width=40,height=40]"\newline
		
\end{itemize}

\subsection{CollisionWithRobot\_IRÁNYVÁLTOZTATÁS\_TESZT}
\begin{itemize}
	\item Leírás: Ez a teszt a KeyPressed(int e) függvény teszteléséért felelős.
			Létrehoz 2 robotot.
			Kiirja az adataikat  és meghivja a KeyPressed(int e)  függvényt VK\_D és VK\_LEFT paraméterekkel majd meghívjuk a move-ot és újból ki listáza őket. A várt eredmény, hogy 180 fokban elfordultak és úgy léptek.			\newline
	\item Ellenőrzött funkcionalitás, várható hibahelyek: Azt ellenörizzük hogy sikeressen ki tudta e számolni a függvény az új koordinátákat.
		Várható hiba hogy rosszul modosítja a cél koordinátákat vagy az alpha szöget. 
		Ezt onnan látjuk, hogy nem a várt koordinátákat látjuk a nextx, nexty, alpha tagváltozókban miután másodjára is kilistáztuk őket vagy mikor a keypressed kiirja.
	\item Bemeneti nyelv :
		\begin{itemize}
		\item Robot(500,500)
		\item Robot(600,400)
		\item listRobots
		\item Keypressed(Keyevent.VK\_A,180)
		\item Keypressed(Keyevent.VK\_LEFT,180)
		\item move
		\item listRobots
		\end{itemize}
	
	\item Elvárt kimenet: \\
		"Robot [id=0,  slowed=1,oiled=false, x=500,y=500, 
		\\numGlue=3,numOil=3,nextx=500,
		\\nexty=600,alpha=1.57,width=40,height=40]"\newline
		"Robot [id=1,  slowed=1,oiled=false, x=600,y=400, 
		\\numGlue=3,numOil=3,nextx=600,
		\\nexty=500,alpha=1.57,width=40,height=40]"\newline
		
		"nextx ,nexty modified to:500,400"\\
		"nextx ,nexty modified to:600,300"\\
	\comment{Ne felejtsd, hogy a keypressed állítja az alpha-t is! Illetve a Robotok magukba fognak ??miért???? ugrani(500,400)->(500,400), (600,300)->(600,300). by Don}\\
		"Robot [id=0,  slowed=1,oiled=false, x=500,y=400, 
		\\numGlue=3,numOil=3,nextx=500,
		\\nexty=400,alpha=4.71238898,width=40,height=40]"\newline
		"Robot [id=1,  slowed=1,oiled=false, x=600,y=300, 
		\\numGlue=3,numOil=3,nextx=600,
		\\nexty=300,alpha=4.71238898,width=40,height=40]"\newline
\end{itemize}

\subsection{robotOutsideOfMap\_A.PÁLYÁRÓL.LEESETT\_TESZT}
\begin{itemize}
	\item Leírás: Ez a teszt a robotOutsideOfMap(Robot r) függvény teszteléséért felelős.
			Létrehoz 1 robotot majd beállítja, úgy hogy kiugorjon a pályáról .
			Kiirja az adatait  és meghivja a move függvényt ezt követően pedig a robotOutsideOfMap függvényt és leellenőrizzük hogy a listában maradt-e a robot.	\newline
	\item Ellenőrzött funkcionalitás, várható hibahelyek: Azt ellenörizzük hogy sikeressen össze tudta-e vetni a hitboxokat a függvény az ugrás végeztével .
	Várható hiba ha hamissal tér vissza a függvény, amiből látjuk, hogy rossz a hitbox létrehozása  az ugrást követően.
	\item Bemeneti nyelv :
		\begin{itemize}
\item MapBuilder map
		\item Robot(400,500)
		\item listRobots
		\item Keypressed(Keyevent.VK\_A,90)
		\item move
		\item listRobots
		\end{itemize}

	\item Elvárt kimenet: \\
		"Robot [id=0,  slowed=1,oiled=false, x=400,y=500, 
		\\numGlue=3,numOil=3,nextx=400,
		\\nexty=400,alpha=1.57,width=40,height=40]"\newline
		
		"nextx ,nexty modified to:400,400"\newline
	
		"Robot [id=0,  slowed=1,oiled=false, x=400,y=500, 
		\\numGlue=3,numOil=3,nextx=400,
		\\nexty=400,alpha=1.57,width=40,height=40]"\newline
		 
		 "there was a collision between this: "Robot [id=0,  slowed=1,oiled=false, x=500,y=400, 
		\\numGlue=3,numOil=3,nextx=500,
		\\nexty=400,alpha=1.57,width=40,height=40]"\newline
		and this:
		"Map"\newline
		
	    "-"\comment{Nincs a robot a listában.}\newline 
\end{itemize}

\subsection{robotOutsideOfMap\_NEM.ESETT.LE.PÁLYÁRÓL\_TESZT}
\begin{itemize}
	\item Leírás: Ez a teszt a robotOutsideOfMap(Robot r) függvény teszteléséért felelős.
			Létrehoz 1 robotot majd beállítja, úgy hogy ne ugorjon ki a pályáról .
			Kiirja az adatait  és meghivja a move függvényt ezt követően pedig a robotOutsideOfMap függvényt és leellenőrizzük hogy a listában maradt-e             a robot.\newline
	\item Ellenőrzött funkcionalitás, várható hibahelyek: Azt ellenörizzük hogy sikeressen össze tudta-e vetni a hitboxokat a függvény az ugrás                     végeztével .
	        Várható hiba ha igazzal tér vissza a függvény, amiből látjuk, hogy rossz a hitbox létrehozása  az ugrást követően.
	\item Bemeneti nyelv :
		\begin{itemize}
\item MapBuilder map
		\item Robot(400,500)
		\item listRobots
		\item Keypressed(Keyevent.VK\_D,90)
		\item move
		\item listRobots
		\end{itemize}

	\item Elvárt kimenet: \\
		"Robot [id=0,  slowed=1,oiled=false, x=400,y=500, 
		\\numGlue=3,numOil=3,nextx=400,
		\\nexty=600,alpha=1.57,width=40,height=40]"\newline
		
		"nextx ,nexty modified to:400,600"\newline
		
		"-"\comment{Nem történt ütközés a robot és a pálya között}\newline
	
		 "Robot [id=0,  slowed=1,oiled=false, x=400,y=500, 
		\\numGlue=3,numOil=3,nextx=400,
		\\nexty=600,alpha=1.57,width=40,height=40]"\newline
	
\end{itemize}


\subsection{checkpointSearch\_CHEICKPOINTONBA.UGRÁS\_TESZT}
\begin{itemize}
	\item Leírás: Ez a teszt a checkpointSearch() függvény teszteléséért felelős.
			Létrehoz 1 robotot majd beállítja, úgy hogy ugorjon bele a checkpointba.
			Kiirja az adatait  és meghivja a move függvényt ezt követően pedig a checkpointSearch függvényt és leellenőrizzük hogy van-e a két hitboxnak metszette.\newline
	\item Ellenőrzött funkcionalitás, várható hibahelyek: Azt ellenörizzük hogy sikeressen össze tudta-e vetni a hitboxokat a függvény az ugrás                     végeztével .
	        Várható hiba ha hamissal tér vissza a függvény, amiből látjuk, hogy rossz a hitbox létrehozása  az ugrást követően.
	\item Bemeneti nyelv :
		\begin{itemize}
\item MapBuilder map
        \item MapBuilder Map
        \item setCheckpoints 100,180
		\item Robot(100,100)
		\item listCheckpoints
		\item listRobots
		\item Keypressed(Keyevent.VK\_D,90)
		\item move
		\item listRobots
		\end{itemize}

	\item Elvárt kimenet: \\
	    "Checkpoint: 100,180"\newline
	    
		"Robot [id=0,  slowed=1,oiled=false, x=100,y=100, 
		\\numGlue=3,numOil=3,nextx=100,
		\\nexty=200,alpha=1.57,width=40,height=40]"\newline
		
		"nextx ,nexty modified to:100,200"\newline
		
		 "there was a collision between this: Robot [id=0,  slowed=1,oiled=false, x=100,y=200, 
		\\numGlue=3,numOil=3,nextx=100,
		\\nexty=200,alpha=1.57,width=40,height=40]
		\\and this: 
		  Checkpoint: 100,180"\newline
	
		 "Robot [id=0,  slowed=1,oiled=false, x=100,y=200, 
		\\numGlue=3,numOil=3,nextx=100,
		\\nexty=200,alpha=1.57,width=40,height=40]"\newline
	
\end{itemize}

\subsection{checkpointSearch\_CHEICKPOINTONBA.NEM.UGRÁS\_TESZT}
\begin{itemize}
	\item Leírás: Ez a teszt a checkpointSearch() függvény teszteléséért felelős.
			Létrehoz 1 robotot majd beállítja, úgy hogy ne ugorjon bele a checkpointba.
			Kiirja az adatait  és meghivja a move függvényt ezt követően pedig a checkpointSearch függvényt és leellenőrizzük hogy van-e a két hitboxnak metszette.\newline
	\item Ellenőrzött funkcionalitás, várható hibahelyek: Azt ellenörizzük hogy sikeressen össze tudta-e vetni a hitboxokat a függvény az ugrás                     végeztével .
	        Várható hiba ha igazzal tér vissza a függvény, amiből látjuk, hogy rossz a hitbox létrehozása  az ugrást követően.
	\item Bemeneti nyelv :
		\begin{itemize}
\item MapBuilder map
        \item MapBuilder Map
        \item setCheckpoints 100,180
		\item Robot(100,300)
		\item listCheckpoints
		\item listRobots
		\item Keypressed(Keyevent.VK\_D,90)
		\item move
		\item listRobots
		\end{itemize}

	\item Elvárt kimenet: \\
	    "Checkpoint: 100,180"\newline
	    
		"Robot [id=0,  slowed=1,oiled=false, x=100,y=300, 
		\\numGlue=3,numOil=3,nextx=100,
		\\nexty=400,alpha=1.57,width=40,height=40]"\newline
		
		"nextx ,nexty modified to:100,400"\newline
		
		 "-"\comment{Nincs chechpoint}\newline
	
		 "Robot [id=0,  slowed=1,oiled=false, x=100,y=400, 
		\\numGlue=3,numOil=3,nextx=100,
		\\nexty=400,alpha=1.57,width=40,height=40]"\newline
	
\end{itemize}






\subsection{Teszteset2}
\begin{itemize}
\item Leírás\newline
\comment{szöveges leírás, kb. 1-5 mondat.}
\item Ellenőrzött funkcionalitás, várható hibahelyek
\item Bemenet\newline
\comment{a proto bemeneti nyelvén megadva (lásd előző anyag)}
\item Elvárt kimenet\newline
\comment{a proto kimeneti nyelvén megadva (lásd előző anyag)}
\end{itemize}

\section{A tesztelést támogató programok tervei}
\comment{A tesztadatok előállítására, a tesztek eredményeinek kiértékelésére szolgáló segédprogramok részletes terveit kell elkészíteni.}


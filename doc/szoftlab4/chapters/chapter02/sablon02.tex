% Szglab4
% ===========================================================================
%
\chapter{Követelmény, projekt, funkcionalitás}

\thispagestyle{fancy}

\section{Bevezetés}

\subsection{Cél}

A project követelményeinek, alapvető felépítésének és funkcionalitásának ismertetése, ezek segítségével a fejlesztési folyamatok majd a végleges program felépítésének megtervezése. Fejlesztés közben ezeket figyelembe kell majd venni és tőlük eltérni nem szabad.

\comment{A dokumentum célja.}

\subsection{Szakterület}
A project a Szoftver laboratórium 4. tantárgy feladatának megoldására született. Az elkészítendő szoftver egy számítógépes játék, nem kimondottan egy szakterület részére készül, inkább szórakoztatás céljából.

\comment{A kialakítandó szoftver milyen területen használható, milyen célra.}

\subsection{Definíciók, rövidítések}
\comment{A dokumentumban használt definíciók, rövidítések magyarázata.}

\subsection {Hivatkozások}
Szoftver Labor 4 - \url{https://www.iit.bme.hu/~szoftlab4/}

\comment{A dokumentumban használt anyagok, web-oldalak felsorolása}

\subsection{Összefoglalás}
A dokumentum további részeiben található:
\begin{itemize}
\item 2.2 Áttekintés: A project terveinek, funkcióinak áttekintése, a felhasználók lehetőségeinek áttekintése
\item 2.3 Követelmények: A project során elvárt követelmények kidolgozása, külön kitérve funkcionális, erőforrás, átadással kapcsolatos és egyéb nem funkcionális követelményekre.
\item 2.4 Lényeges use-case-ek felsorolása, use-case diagram
\item 2.5 Szótár: A project során bevezetett fogalmak körülírása.
\item 2.6 Project terv: A project résztvevőinek feladatkörei és határidők részletes kifejtése
\item 2.7 Napló: Az elvégzett feladatok és ráfordított idő felsorolása
\end{itemize}

\comment{A dokumentum további részeinek rövid ismertetése}

\pagebreak
\section{Áttekintés}
\subsection{Általános áttekintés}
\begin{itemize}
\item játékmotor
\item director
\item HUD
\item játék menü
\item garfikus felület
\item pályák
\item robotok
\end{itemize}
A szoftver fontos eleme a Játékmotor. Ez teremti meg a kezdeti feltételeket és irányítja a játék folyását. Ez fogja ellenőrizni, hogy a robotok leesnek-e a pályáról, ha olajfoltba lépnek letiltja az irányítást a játékos oldalon, illetve ha ragacsba lép egy játékos, akkor lelassuljon. A Játékmotor indulása előtt lekéri a Pályák alrendszertől a kiválasztott pályát.
A director alrendszer feladata az idő léptetése, másodpercenként ugranak a robotok ezt ütemezi.
A HUD alrendszer hivatott jelezni merre fogunk elmozdulni a robotunkkal és jelzi, ha van kivethető ragacs vagy olajfolt a robotnál.
A játék menü alrendszer feladata a játék során a megfelelő parancsra megállítani a játékot és a beállítások (pl.: grafikai, hang) állíthatósága.
A grafikus felület feladata, hogy kapcsolatot létesítsen a játékossal vizuális információkkal és a játékos ezen keresztül irányíthatja a játékmotort.
A Pályák alrendszer betölti a Pályafájlokból a pályákat, tárolja, majd azokat a Játékmotor rendelkezésére bocsájtja.
A Robotok alrendszer betölti a robotokat a Robotfájlokból, tárolja azokat, majd a Játékmotor rendelkezésére bocsájtja.
Hálózatot a szoftver nem igényel. Háttértáron pedig a program .jar archívumainak, a kirajzoláshoz szükséges pálya és robot képeknek, a pályákat tartalmazó fájloknak szükséges helyet biztosítani. Ezeken kívül nincs szükség futás idejű tárhelyigényre. 

\comment{A kialakítandó szoftver legmagasabb szintű architekturális képe. A fontosabb alrendszerek felsorolása, a közöttük kialakítandó interfészek lényege, a felhasználói kapcsolatok alapja. Esetleges hálózati és adattárolási elvárások.}

\subsection{Funkciók}
\comment{A feladat kb. 4000 karakteres (kb 1,5 oldal) részletezettségű magyar nyelvű leírása. Nem szerepelhetnek informatikai kifejezések.}

\subsection{Felhasználók}
A játék 2-4 játékos módban indíthat. A játékosnak semmiféle előképzettségre nincs szüksége, a játék menete gyorsan elsajátítható. A játékban nem lesz pályaszerkesztő. A játékot nem lehet elmenteni, mindig újból kell kezdeni.
\comment{A felhasználók jellemzői, tulajdonságai}

\subsection{Korlátozások}
A BME IIT Szoftver Laboratórium tantárgy oktatói (a megrendelők) által kiírt specifikáció megköveteli, hogy a program a HSZK gépein a beadások alkalmával futtathatók legyenek. Ezekre a gépekre már előre feltelepített JRE 1.7-es környezet elérhető, ezért csak a JDK 1.7-es verziójaban már létező függvények, osztályok használhatóak a kód megírása során.
\comment{Az elkészítendő szoftverre vonatkozó – általában nem funkcionális - előírások, korlátozások.}

\subsection{Feltételezések, kapcsolatok}
\comment{A dokumentumban használt anyagok, web-oldalak felsorolása}

\section{Követelmények}
\subsection{Funkcionális követelmények}
Prioritások:
\begin{itemize}
\item alapvető
\item fontos
\item opcionális
\end{itemize}
Források:
\begin{itemize}
\item specifikáció
\item csapat
\item konzulens
\end{itemize}

\comment{Az alábbi táblázat kitöltésével készítendő. Dolgozzon ki követelmény azonosító rendszert! Az ellenőrzés módja szokásosan bemutatás és/vagy kiértékelés. Prioritás lehet alapvető, fontos, opcionális. Az alapvető követelmények nem teljesítése végzetes. Forrás alatt a követelményt előíró anyagot, szervezetet kell érteni. Esetünkben forrás lehet maga a csapat is, mikor ő talál ki követelményt. Use-case-ek alatt az adott követelményt megvalósító használati esete(ke)t kell megadni.}

% Azonosító, Leírás, Ellenőrzés, Prioritás, Forrás, Use-case, Komment
\begin{longtable}{| l | l | l | l | l | l | l |}
\hline
\textbf{Azonosító}   & \textbf{Leírás} & \textbf{Ellenőrzés} & \textbf{Prioritás} & \textbf{Forrás} & \textbf{Use-case} & \textbf{Komment} \tabularnewline
\hline 1.01 & Játék menü & & & & &
\hline 1.02 & Beállítások kezelése & & & & &
\hline 1.03 & Játék elindítása & & & & &
\hline 1.04 & Pályáról leesők kiesnek & & & & &
\hline 1.05 & Előre elkészített versenypálya & & & & &
\hline 1.06 & \vtop{\hbox{\strut Robotok a kezdőpozíciójukból}\hbox{\strut indulnak}}  & & & & &
\hline 1.07 &\vtop{\hbox{\strut Pályán vannak olajfoltok és}\hbox{\strut ragacsfoltok}}   & & & & &
\hline 1.08 &\vtop{\hbox{\strut Robotok fel vannak szerelve olaj}\hbox{\strut  és ragacskészlettel}}  & & & & &
\tabularnewline
\hline
\end{longtable}

\subsection{Erőforrásokkal kapcsolatos követelmények}

\comment{A szoftver fejlesztésével és használatával kapcsolatos számítógépes, hardveres, alapszoftveres és egyéb architekturális és logisztikai követelmények}

% Azonosító, Leírás, Ellenőrzés, Prioritás, Forrás, Komment
\begin{longtable}{| l | l | l | l | l | l |}
\hline
\textbf{Azonosító}   & \textbf{Leírás} & \textbf{Ellenőrzés} & \textbf{Prioritás} & \textbf{Forrás} & \textbf{Komment} \tabularnewline
\hline
\hline 2.01 & JRE  &  &  &  & Java IDE  \tabularnewline
\hline 2.02 & Eclipse &  &  &  & Jav IDE \tabularnewline
\hline 2.03 & Monitor &  &  &  &  \tabularnewline
\hline 2.04 & Egér &  &  &  &  \tabularnewline
\hline 2.05 & ShareLatex &  &  &  & online LaTeX editor  \tabularnewline
\hline 2.06 & Git &  &  &  & Elosztott verziókezelő \tabularnewline
\hline 2.07 & GitHub account &  &  &  & Git tárhely \tabularnewline
\hline
\end{longtable}


\subsection{Átadással kapcsolatos követelmények}
\comment{A szoftver átadásával, telepítésével, üzembe helyezésével kapcsolatos követelmények}

% Azonosító, Leírás, Ellenőrzés, Prioritás, Forrás, Komment
\begin{longtable}{| l | l | l | l | l | l |}
\hline
\textbf{Azonosító}   & \textbf{Leírás} & \textbf{Ellenőrzés} & \textbf{Prioritás} & \textbf{Forrás} & \textbf{Komment} \tabularnewline
\hline\hline
... & ... & ... & ... & ... & ... \tabularnewline
\hline
\end{longtable}

\subsection{Egyéb nem funkcionális követelmények}
\comment{A biztonsággal, hordozhatósággal, megbízhatósággal, tesztelhetőséggel, a felhasználóval kapcsolatos követelmények}

% Azonosító, Leírás, Ellenőrzés, Prioritás, Forrás, Komment
\begin{longtable}{| l | l | l | l | l | l |}
\hline
\textbf{Azonosító}   & \textbf{Leírás} & \textbf{Ellenőrzés} & \textbf{Prioritás} & \textbf{Forrás} & \textbf{Komment} \tabularnewline
\hline\hline
... & ... & ... & ... & ... & ... \tabularnewline
\hline
\end{longtable}


\section{Lényeges use-case-ek}
\comment{A 2.3.1-ben felsorolt követelmények közül az alapvető és fontos követelményekhez tartozó használati esetek megadása az alábbi táblázatos formában.}
\subsection{Use-case leírások}

\comment{Minden use-case-hez külön}

\usecase{...}{...}{...}{...}

\usecase{...}{...}{...}{...}

\section{Szótár}
\comment{A szótár a követelmények alapján készítendő fejezet. Egy szótári bejegyzés definiálásához csak más szótári bejegyzések és köznapi – a feladattól független – fogalmak használhatók fel. A szótár mérete kb. 1-2 oldal legyen.}

\section{Projekt terv}
\subsection{Csapat}
A csapat 5 főből áll. A következő táblázat a személyes preferenciákat tartalmazza, a feladatokat úgy osztjuk ki, hogy közel azonos nehézségűek legyenek, miközben ezeket a preferenciákat szem előtt tartjuk.
\comment{Tartalmaznia kell a projekt végrehajtásának lépéseit, a lépések, eredmények határidejét, az egyes feladatok elvégzéséért felelős személyek nevét és beosztását, a szükséges erőforrásokat, stb. Meg kell adni a csoportmunkát támogató eszközöket, a választott technikákat! Definiálni kell, hogy hogyan történik a dokumentumok és a forráskód megosztása!}


